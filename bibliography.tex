\begin{thebibliography}{99}


\bibitem{AllenJacobson2023}
L. M. Allen-Jacobson, A. W. Jones, A. J. Mercer, S. X. Cadrin, B. Galuardi, D. Christel, A. Silva, A. Lipsky, and J. B. Haugen, ``Evaluating Potential Impacts of Offshore Wind Development on Fishing Operations by Comparing Fine- and Coarse-Scale Fishery-Dependent Data,'' \textit{Marine and Coastal Fisheries}, vol. 15, no. 1, p. e10233, 2023. [Online]. Available: https://doi.org/10.1002/mcf2.10233. [Accessed: 27-Nov-2025].

% --- 第1章 序論:背景(世界・日本の動向) ---

\bibitem{IRENA_2025}
% 【背景】世界の再生可能エネルギー容量の増加統計(2015-2025年の急増を示す根拠)
International Renewable Energy Agency (IRENA), ``Renewable Capacity Statistics 2023,'' Abu Dhabi, 2023. [Online]. Available: https://www.irena.org/Publications/2025/Mar/Renewable-capacity-statistics-2025. [Accessed: 26-Nov-2025].

\bibitem{Energy_Plan_6th}
% 【背景】日本の「第6次エネルギー基本計画」。再エネ比率目標(36-38%)や洋上風力の主力電源化の根拠。
経済産業省 資源エネルギー庁, ``第6次エネルギー基本計画,'' 2021年10月22日閣議決定. [Online]. Available: https://www.enecho.meti.go.jp/category/others/basic\_plan/pdf/20211022\_01.pdf. [Accessed: 27-Nov-2025].

\bibitem{Green_Growth_Strategy}
% 【背景】「2050年カーボンニュートラル宣言」とそれに伴う洋上風力産業の成長戦略。
経済産業省, ``2050年カーボンニュートラルに伴うグリーン成長戦略,'' 2020年12月. [Online]. Available: https://www.meti.go.jp/policy/energy\_environment/global\_warming/ggs/pdf/green\_honbun.pdf. [Accessed: 27-Nov-2025].

\bibitem{Paris_Agreement}
% 【背景】世界的な脱炭素の流れの起点となった「パリ協定」。気温上昇抑制目標など。
United Nations Framework Convention on Climate Change (UNFCCC), ``The Paris Agreement,'' Dec. 2015. [Online]. Available: https://unfccc.int/sites/default/files/english\_paris\_agreement.pdf. [Accessed: 27-Nov-2025].

\bibitem{JWPA_Stats_2024}
% 【背景】日本国内における風力発電の最新の導入状況・累積導入量。
日本風力発電協会 (JWPA), ``2024年12月末時点日本の風力発電の累積導入量,'' 2025年2月18日. [Online]. Available: https://jwpa.jp/information/11062/. [Accessed: 27-Nov-2025].

% --- 第1章・第2章:先行事例・社会的受容性(問題の所在) ---

\bibitem{akita} 
% 【先行事例】秋田県における漁業者との合意形成の難しさ、補償交渉の課題などを指摘した事例研究。
山口健介・田嶋智・渡部熙・城山英明, 「我が国の洋上風力事業における漁業者との合意形成: 秋田県男鹿市,潟上市及び秋田市沖における事例と政策提言」, 日本海洋政策学会誌, 13号, pp65-81, 2023.

\bibitem{Aomori}
% 【先行事例】青森県の漁業者へのアンケート調査。「不明確な影響」への懸念が反対理由であることを示す。
桐原慎二,
``洋上風力発電に対する漁業者の意向―青森県の漁業者を対象としたアンケート調査から―,''
水産工学, vol.~57, no.~2, pp.~65--77, 2021.
[Online]. Available: https://doi.org/10.18903/fisheng.57.2\_65.
[Accessed: 27-Nov-2025].

\bibitem{Hooper2015}
% 【海外事例・受容性】英国のカニ・エビ漁業者の認識。物理的共存が可能でも、操業エリア喪失への懸念が強いことを指摘。
T.~Hooper, M.~Ashley, and M.~Austen, 
``Perceptions of fishers and developers on the co-location of offshore wind farms and decapod fisheries in the UK,'' 
Marine Policy, vol.~61, pp.~16--22, 2015. [Online]. Available: https://doi.org/10.1016/j.marpol.2015.06.031. [Accessed: 27-Nov-2025].

\bibitem{Reilly2015}
% 【海外事例・受容性】アイルランドの漁業者の意識調査。開発プロセスへの不信感などを指摘。
K.~Reilly, A.~M. O'Hagan, and G.~Dalton, 
``Attitudes and perceptions of fishermen on the island of Ireland towards the development of marine renewable energy projects,'' 
Marine Policy, vol.~58, pp.~88--97, 2015. [Online]. Available: https://doi.org/10.1016/j.marpol.2015.04.001. [Accessed: 27-Nov-2025].

% --- 第2章:経済影響評価・統計手法(マクロ分析の限界) ---

\bibitem{Shimada2022}
% 【重要・経済評価】日本の洋上風力と漁獲量の関係を統計的(SCM)に分析。「マクロ統計では有意な影響なし」としたが、局所的影響は見えていないという対比に使う。
H. Shimada, K. Asano, Y. Nagai, and A. Ozawa, ``Assessing the Impact of Offshore Wind Power Deployment on Fishery: A Synthetic Control Approach,'' \textit{Environmental and Resource Economics}, vol. 83, pp. 791--829, 2022.

% --- 第2章:環境・生態系への影響 ---

\bibitem{Boehlert2010}
% 【環境影響】海洋再生可能エネルギーが生態系に与えるストレッサー(騒音、電磁界等)の総説。
G. W. Boehlert and A. B. Gill, ``Environmental and ecological effects of ocean renewable energy development: a current synthesis,'' \textit{Oceanography}, vol. 23, no. 2, pp. 68--81, 2010.

\bibitem{Bailey2014}
% 【環境影響】建設時の杭打ち音(騒音)が海洋生物に与える回避行動などの影響。
H. Bailey, K. L. Brookes, and P. M. Thompson, ``Assessing environmental impacts of offshore wind farms: lessons learned and recommendations for the future,'' \textit{Aquatic Biosystems}, vol. 10, no. 1, pp. 1--13, 2014.

\bibitem{akamatsu_2018}
% 【環境影響】水中騒音のアセスメント手法に関する国内の技術資料。
赤松 友成, ``海洋生物の新しいアセスメント手法 ~水中騒音による影響を中心として~,'' 平成30年度 環境アセスメント技術講習会(仙台会場) 資料2, 2018年. [Online]. Available: \url{https://assess.env.go.jp/files/4_kentou/4-2_training/h30/sendai_h30_02.pdf}. [Accessed: Nov. 27, 2025].

\bibitem{matsumae_council_2024}
北海道松前沖における協議会, ``北海道松前沖における協議会意見とりまとめ,'' 2024年7月. [Online]. Available: \url{https://www.mlit.go.jp/kowan/content/001757978.pdf}. [Accessed: Dec. 15, 2025].

% --- 第2章:社会的受容性(信頼・プロセス) ---

\bibitem{Alexander2013}
% 【受容性】スコットランドの漁業者調査。「信頼の欠如」が紛争の原因であると指摘。
K. A. Alexander, T. A. Wilding, and J. J. Heymans, ``Attitudes of Scottish fishers towards marine renewable energy,'' \textit{Marine Policy}, vol. 37, pp. 239--244, 2013.


% --- 第2章:その他の定量評価事例 ---

\bibitem{Jensen2018}
% 【他分野事例】風力発電が地価に与える影響分析。統計的因果推論の事例として紹介。
C. U. Jensen, T. E. Panduro, T. H. Lundhede, A. S. E. Nielsen, M. Dalsgaard, and B. J. Thorsen, ``The impact of on-shore and off-shore wind turbine farms on property prices,'' \textit{Energy Policy}, vol. 116, pp. 50--59, 2018.

\bibitem{Ando2015}
% 【手法参考】原発立地の影響をSCM(合成コントロール法)で分析した研究。Shimada et al. (2022) のベースとなった手法。
M. Ando, ``Dreams of urbanization: quantitative case studies on the local impacts of nuclear power facilities using the synthetic control method,'' \textit{Journal of Urban Economics}, vol. 85, pp. 68--85, 2015.

% --- 第2章:制度・ポジティブな影響・課題 ---

\bibitem{meti_yojo_2023}
% 【制度】洋上風力に係る環境影響評価の現状と課題に関する政府資料。
経済産業省 資源エネルギー庁, 国土交通省 港湾局, ``洋上風力発電に係る環境影響評価について,'' 第18回 洋上風力促進ワーキンググループ 資料3, 2023年1月. [Online]. Available: \url{https://www.meti.go.jp/shingikai/enecho/denryoku_gas/saisei_kano/yojo_furyoku/pdf/018_03_00.pdf}. [Accessed: Nov. 27, 2025].

\bibitem{Lindeboom2011}
% 【ポジティブ影響】オランダの事例。風車基礎が人工魚礁となり、生物多様性が増した(Reef Effect)という報告。
H. J. Lindeboom \textit{et al.}, ``Short-term ecological effects of an offshore wind farm in the Dutch coastal zone; a compilation,'' \textit{Environmental Research Letters}, vol. 6, no. 3, 035101, 2011.

\bibitem{moe_offshore_2022}
% 【課題】環境省報告書。既存のアセスメント手法だけでは、漁業影響を十分に測れない可能性があるという指摘。
環境省 総合環境政策局, ``洋上風力発電に係る新たな環境アセスメント制度の在り方について(報告書),'' 2022年3月. [Online]. Available: \url{https://assess.env.go.jp/files/0_db/seika/1055_03/report.pdf}. [Accessed: Nov. 27, 2025].

% --- 第2章・手法:ICT活用 ---

\bibitem{Wada2011}
佐野 稔, 前田 圭司, 高柳 志朗, 和田 雅昭, 畑中 勝守, 本前 伸一, 菊池 肇, 宮下 和士, ``小漁業情報を用いた北海道北部沿岸域におけるマナマコの資源量推定,'' \textit{日本水産学会誌}, vol. 77, no. 6, pp. 999--1007, 2011. [Online]. Available: https://doi.org/10.2331/suisan.77.999. [Accessed: 27-Nov-2025].

\bibitem{maglog}
祐川 雅治, 石井紘翔, 伊禮莉, 小笠原海都, 小田祐希, 金田凌弥, 川浪昂矢, 神田空也, 齊藤良輝, ``公立はこだて未来大学 2024 年度 システム情報科学実習グループ報告書,'' 公立はこだて未来大学 プロジェクト学習, 2024. [Online]. Available: https://www.fun.ac.jp/wp-content/uploads/2025/05/group11.pdf.
神田空也 Soraya Kanda
齊藤良輝 Yoshiki Saito ``公立はこだて未来大学 2024 年度 システム情報科学実習グループ報告書,'' 公立はこだて未来大学 プロジェクト学習, 2024. [Online]. Available: https://www.fun.ac.jp/wp-content/uploads/2025/05/group11.pdf.
[Accessed: 27-Nov-2025].

\bibitem{Zhu2021}
朱 夢瑶, ``水産資源モニタリング手法の現状と今後の展望,'' \textit{OPRI Perspectives}, no. 21, pp. 1--10, 2021. [Online]. Available: https://www.spf.org/global-data/opri/perspectives/prsp\_021\_2021\_zhu.pdf.[Accessed: 27-Nov-2025].

\bibitem{Jenks1967}
G. F. Jenks, ``The Data Model Concept in Statistical Mapping,'' \textit{International Yearbook of Cartography}, vol. 7, pp. 186--190, 1967.

\end{thebibliography}