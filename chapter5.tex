\chapter{考察}
本章では、第4章で得られた可視化結果と操業効率分析に基づき、(1) 松前沖のマグロ延縄漁業における空間利用の特徴、(2) 促進区域を含む海域利用調整に向けた示唆、(3) 影響評価のためのベースラインとしての有効性について考察する。特に、洋上風車建設に伴う移動コストの増加が漁業経営に与える潜在的なインパクトを定量的に評価し、具体的な共存策を提案する。

\section{空間利用実態の解釈とデータ統合の意義}

\subsection{操業日誌ベースの可視化が捉える「操業範囲」}
操業日誌データに基づく利用頻度分布(図4.1)は、漁獲の有無に依存せず、出港から帰港までの全航跡を網羅している。そのため、漁場だけでなく、魚群探索を含む実運用上の操業範囲を直接的に示しているといえる。
本研究の結果、2024年・2025年ともに松前町沿岸から北西方向に高頻度の利用域が形成されており、当該海域が中核的な漁場として恒常的に利用されていることが示唆された。一方で、2025年には2024年と比較し低い利用頻度が広域(特に南西側)に拡散する傾向が見られた。これは、4.3節で確認された「弱い正の相関(移動するほど漁獲が増える)」とも整合しており、2025年はマグロの回遊ルートが、例年よりも沿岸から離れた海域にシフトしていた可能性が考えられる。この場合、近場での操業に見切りをつけ、沖合で操業をした漁船のみが漁獲を確保できたことで、距離と漁獲の相関が確認されたと解釈できる。

\subsection{水揚げ記録ベースとの比較が示す評価上の注意点}
水揚げ記録ベースの分布(図4.2)は、漁獲実績のある日の操業のみを抽出しているため、対象魚種であるクロマグロの来遊している海域を把握するには適している。一方で、水揚げ記録ベースの分布では「探索したが漁獲に至らなかった海域」の情報が欠落するリスクがある。本研究の比較結果(図4.3、図4.4)では操業日誌ベースと水揚げ記録ベースの両手法の分布形状に大きな差異は見られなかったものの、この点は評価上の注意点として考慮すべきである。
洋上風力発電の影響評価においては、漁獲が高い場所だけでなく、そこに至るまでの「探索・移動プロセス」も保護されるべき対象である。したがって、合意形成の場においては、水揚げ記録だけでなく、本研究で提案した操業日誌ベースの可視化結果を併用することが望ましい。

\section{促進区域との関係と海域利用調整への示唆}

\subsection{促進区域と漁船航行経路の空間的関係}
促進区域との空間的重複の評価(図4.5)において、緑の点で示した投縄開始地点は促進区域から少なくとも2nm離れており、直接的な漁場重複は確認されなかった。しかし、海域利用調整において重要なのは実際に漁獲をしている海域だけではない。
操業日誌ベースの利用頻度分布は、主要漁場(北西沖)へ向かうための「漁場への航路」も可視化している。促進区域は沿岸に沿って帯状に指定されており、港から漁場へ向かう漁船にとって、この区域が操業の障壁となる可能性がある。
もし洋上風車そのものや工事を行う船舶によって航路が制限されれば、漁船は区域を迂回する必要が生じる。これは、単なる移動時間の増加だけでなく、次節で述べる燃油コストの増加といったリスクにも影響を与える。

\subsection{共存策に向けた運用的な提案}
本研究の分析結果は、促進区域の評価において、単に漁場の位置的な重複を避けるだけでなく、港から漁場への航路を確保することの重要性を示唆している。これらを踏まえ、漁業と発電事業が共存するための具体的な運用案として、以下の3点を提案する。

\begin{enumerate}
    \item \textbf{漁場へ航路の確保}:
    促進区域の配置が、複数ある港から漁場へ向かう最短ルートを遮断しないよう配慮すること。具体的には、区域内またはその周辺において、漁船が安全かつ効率的に航行可能なルートを明確に設定する必要がある。
    
    \item \textbf{検証された既存システムによる情報共有}:
    本研究のデータ検証(3.2.3項等)により、操業日誌アプリ「マグログ」の利用が定着し、地域漁業における情報インフラとして機能していることが確認された。今後は、本システムを単なる記録ツールとしてだけでなく、洋上風力発電事業との共存プラットフォームへと拡張することも検討するべきである。具体的には、工事を行う船舶の位置情報や海域規制情報をリアルタイムに共有する機能の実装が、今後の沿岸漁業と洋上風力発電の共存に向けた重要な検討課題となる。
    
    \item \textbf{操業効率を考慮した工程の調整}:
    4.3節で明らかになった通り、CPUEおよび燃油効率が高い時期(7月〜8月)は漁業経営にとって極めて重要な期間である。この時期における洋上風車の建設工事や、工事を行う船舶の往来を抑制するなど、漁期の繁閑や効率の変動に応じた柔軟な工程調整が求められる。
\end{enumerate}

\section{操業効率指標からみた経済的影響のリスク評価}

\subsection{移動負担の増加に伴う操業効率への影響試算}
4.3節の分析では、2025年の燃油効率($E_o$)が前年比で悪化(0.852 $\to$ 0.819 kg/L)していることが明らかになった。これは、単位漁獲量を得るために、より多くのエネルギー(燃油)を投入せざるを得なかったことを意味する。

洋上風力発電の導入に伴い、実際にどの程度の「迂回」が発生するかについて、日本海洋センターの指針においては、発電設備や工事を行う船舶の周囲に半径500mの安全水域(Safety Zone)を設定することが推奨されている\cite{JPMAC_Guideline}。
これは、進路上に1つの障害物(工事区画等)が存在する場合、その回避には直径換算で少なくとも1km(1,000m)規模の空間的調整が必要となることを示唆している。

また、Grayら(2016)は、こうした施設による海域占有が漁船の航行可能域を狭める「空間的圧迫(Spatial Squeeze)」を引き起こし、移動時間の延長と燃油コストの増大を招くと指摘している\cite{Gray2016}。米国では18.5kmもの迂回が生じた事例も報告されているが\cite{Samoteskul2014}、本研究では前述の国内指針に基づき、安全水域の直径(1km)を基準とした現実的な迂回シナリオを設定した(表5.2)。

評価に際しては、2025年の操業実績データ(表\ref{tab:calc_basis})をベースラインとし、以下の算出モデルを用いた。

\begin{table}[htbp]
  \centering
  \caption{試算に用いた2025年の操業基礎パラメータ}
  \label{tab:calc_basis}
  \begin{tabular}{lcc}
    \hline
    項目 & 値 & 備考 \\
    \hline
    年間総給油量 ($Q_{total}$) & 72,174 L & 対象船団の合計値 \\
    年間総移動距離 ($D_{total}$) & 3,079.6 nm & GNSS航跡より算出 \\
    年間総航海回数 ($N$) & 185 回 & 操業日誌データより \\
    \textbf{平均燃費係数 ($F_{rate}$)} & \textbf{23.4 L/nm} & $Q_{total} / D_{total}$ \\
    \hline
  \end{tabular}
\end{table}

\subsubsection{算出方法}
1航海あたりの迂回距離を $\Delta d$ (nm) とした時、年間で増加する燃油消費量 $\Delta Q$ (L) は次式で定義される。

\begin{equation}
    \Delta Q = \Delta d \times N \times F_{rate}
\end{equation}

ここで、迂回距離 $\Delta d$ については、日本海洋センターの指針に基づく安全領域(半径500m)\cite{JPMAC_Guideline}を基準として、片道および往復での回避を想定した現実的なシナリオを設定した。

\begin{table}[htbp]
  \centering
  \caption{指針に基づく安全水域回避を想定した燃油消費への影響試算}
  \label{tab:physical_burden}
  \begin{tabular}{lccc}
    \hline
    シナリオ & 想定される回避行動 & 追加距離/航海 & 追加燃油消費量/年 \\
    \hline
    \textbf{現状維持} & - & 0 m & 0 L \\
    \textbf{微小な迂回} & 進路の微修正 & 約 +300 m & 約 +720 L \\
    \textbf{局所的な迂回} & 安全水域1つ分の回避 & 約 +1.0 km & 約 +2,160 L \\
    \textbf{標準的な迂回} & 複数領域または広域回避 & 約 +1.5 km & 約 +3,600 L \\
    \hline
  \end{tabular}
  \begin{flushleft}
    \small
    ※「局所的な迂回」の+1.0kmは、指針\cite{JPMAC_Guideline}で示される安全領域(半径500m)の直径に相当する物理的な最小回避距離として設定した。
    ※追加燃油消費量は、2025年の総給油量(72,174L)を基準に算出。
  \end{flushleft}
\end{table}

表\ref{tab:physical_burden}に示す通り、指針に準拠して安全水域を1つ回避するだけの局所的な迂回(+1.0km)であっても、年間で約2,160Lの燃油が追加消費される計算となる。
これは「漁獲量が変わらない」前提の数値であるが、実際には回避行動による時間のロスが実操業時間の減少を招く可能性もある。したがって、海域利用調整においては、単に通行が可能であるかだけでなく、こうした安全水域回避に伴うエネルギー損失を最小化する航路設定が不可欠である。


\subsection{季節性を踏まえた「海域利用調整カレンダー」の提案}
月別推移(図4.7、図4.8)より、CPUEおよび燃油効率は7月〜8月にピークを迎え、9月以降に低下する傾向が確認された。効率が高い時期は、短い移動・少ない燃油で多くの漁獲が得られる「稼ぎ時」であり、この時期の操業阻害は経済的損失が大きい。
そこで、本研究のデータに基づき、工事やメンテナンス作業が漁業に与える影響を最小化するための「海域利用調整カレンダー案」を表5.2に提案する。

\begin{table}[htbp]
  \centering
  \caption{操業データに基づく時期別・漁業影響リスク評価}
  \label{tab:fishery_risk_assessment}
  % 列をシンプルに3つに絞る。X列で幅を自動調整。
  \begin{tabularx}{\textwidth}{cXc} 
    \hline
    月 & \multicolumn{1}{c}{漁業実態(データ特性)} & \multicolumn{1}{c}{漁業への影響リスク可能性} \\
    \hline
    7月 & \textbf{最盛期}。CPUEおよび燃油効率が最高値を示す。 & \textbf{極めて大} \\
    \hline
    8月 & \textbf{高水準維持}。高い操業効率が維持される。 & \textbf{大}  \\
    \hline
    9月 & \textbf{過渡期}。年変動があるが、7、8月よりも漁獲量は低下傾向。 & \textbf{中}  \\
    \hline
    10月 & \textbf{漁期後半}。一定の操業は継続しているが漁獲量は年変動あり。 & \textbf{中}  \\
    \hline
    11月 & \textbf{漁期後半}。漁獲努力量(出漁回数)が減少する。 & \textbf{小} \\
    \hline
    12月$\sim$ & \textbf{漁期終盤}。対象漁業の操業が最も少ない。 & \textbf{極小} \\
    \hline
  \end{tabularx}
\end{table}
このように、定量的な操業効率データに基づいて「施工の影響が相対的に小さい時期」と「操業効率が高く影響を最小化すべき時期」を明確化することで、事業者と漁業者の双方が納得感のある海域利用調整の枠組みを構築できると考えられる。

\section{ベースラインとしての有効性}

\subsection{ベースラインの意義}
本研究は、洋上風車建設前の操業実態を、位置情報と操業記録の統合により定量化した点に最大の価値がある。これまでは「漁業者の勘」として定性的に語られていた事象(例:夏は近くで釣れる、今年は遠くまで行った等)が、数値データとして裏付けられた。
将来的に建設後の同種データを取得できれば、本研究で構築した2024-2025年のデータをベースライン(比較基準)として、空間利用の変化や効率の悪化を統計的に検証することが可能となる。これは、科学的根拠に基づく順応的な管理(Adaptive Management)を実現するための基礎となる。
