\chapter{結論}

\section{本研究のまとめ}
本研究では、北海道松前沖における洋上風力発電と沿岸漁業の共存に向け、異種データ(GNSS航跡、水揚げ記録、操業日誌アプリ、給油履歴)を統合することで、マグロ延縄漁業の操業実態を「空間」と「効率」の両面から定量的に可視化した。
従来、漁業影響評価は漁場の消失に焦点が当てられがちであったが、本研究のデータ分析は、移動経路と燃油効率・CPUEという新たな評価軸を用いて評価を行った。本研究で得られた主要な知見は以下の3点に集約される。

\begin{enumerate}
    \item \textbf{漁場保全と航路確保の両立課題}:
        3カ年の空間利用分析の結果、マグロ延縄漁業の主要漁場は促進区域外の北西沖に形成されており、発電設備設置予定海域との直接的な重複は確認されなかった。しかし、港から漁場へ至る航路は促進区域を横断しており、洋上風車建設が漁場への物理的な障壁となるリスクが定量的に示された。すなわち、漁場そのものは保全可能であっても、移動経路の確保がなされなければ、実質的な操業機会が損なわれる構造にある。

    \item \textbf{移動コスト増大による経済的脆弱性}:
        CPUEおよび燃油効率の分析により、7月〜8月が高効率な最盛期であることが特定された。一方で、2025年の分析結果は、探索行動により移動距離が増加しても、必ずしも漁獲増に結びつかず、燃油効率が悪化する実態を浮き彫りにした。これは、洋上風車建設に伴う迂回行動が強制された場合、漁獲量が変わらなくとも「燃油コストの増加」によって漁業経営が圧迫されるリスクを示唆している。

    \item \textbf{漁業者からの操業データ提供に基づく共存ベースラインの確立}:
        本研究の最大の特長は、地元漁業者から直接提供された高精細な操業データを用いた点にある。漁業者へのアンケートや水揚げ記録だけでは見えない「漁獲に至らなかった操業」を含めた解析を行ったことで、漁業実態を正確に反映した評価が可能となった。この現場の実態に基づくベースラインの構築が、事業者と漁業者がデータに基づいて議論し、共存を実現するための基盤となるものになる。
\end{enumerate}

\section{社会的・学術的貢献}
本研究の社会的意義は、洋上風力発電と沿岸漁業の共存に向けた議論において、客観的な数値データに基づく判断材料を提示した点にある。
これまで感覚的に示されることの多かった操業への影響や負担を、移動距離・航路や燃油効率という具体的な指標で可視化したことは、事業者と漁業者が具体的な調整(航路や時期の検討、漁業補償額の提案)を行う上で有用な情報となる。

また、学術的には、性質の異なる複数のデータを組み合わせることで、単一のデータだけでは捉えきれない詳細な操業実態を明らかにした点に意義がある。
本研究で用いた、漁業者の記録とVMSからのデータを突き合わせるアプローチは、複雑な利用実態を持つ沿岸域において、実効性のある海域利用調整を行うための有効な手法の一つであると考えられる。

\section{今後の展望}
最後に、本研究の成果を発展させ、共存を実現するために取り組むべき今後の課題を述べる。

\subsection{多魚種・通年データの統合による全体像の解明}
本研究では主要漁業であるマグロ延縄漁業に焦点を当てたが、当該海域ではスルメイカ漁や刺し網漁など、季節や操業形態の異なる多様な漁業が営まれている。海域利用調整の実効性を高めるためには、これら他魚種のデータも同様の手法で可視化し、海域利用の季節的・空間的な全体像を明らかにすることが不可欠である。

\subsection{継続的なモニタリングとBA(Before-After)調査}
本研究で構築した「空間」と「効率」の定量的評価は、あくまで洋上風車建設前(Before)のベースラインである。洋上風力発電が漁業に与える真の影響を検証するためには、今後、工事期間中および運転開始後(After)においても継続的にデータを取得し、本研究の結果と比較を行う「BA(Before-After)調査」を実施する必要がある。
この時系列的な比較検証を通じて初めて、予測された「迂回による燃油効率の悪化」や「漁場アクセスの阻害」が実際にどの程度発生したかを事後評価し、必要に応じた順応的な対策を講じることが可能となる。

\subsection{解析手法のオープンソース化と透明性の確保}
本研究で開発した一連の解析プログラム(異種データの統合、操業効率算出等のアルゴリズム)については、オープンソースソフトウェア(OSS)として公開し、誰でも利用可能な状態にすることを計画している。その公開に向けた整理として、本手法の全体像を図\ref{fig:data_flow}に示す。

本図は、4種類の入力データ(GNSS航跡データ、水揚げ記録、操業日誌データ、給油履歴)がPythonによる統合処理を経て、空間分析と操業効率評価の2軸に展開される過程をまとめたものである。各データは漁船IDと日付をキーとして照合され、移動距離の算出にはHaversineの公式を適用している。
具体的な解析プロセスとして、空間分析ではQGISを用いて内接円の直径が1km(0.539 nm)の六角形グリッドによる集計を行い、利用頻度分布の可視化および促進区域との重複評価を実施する。一方、操業効率評価では、CPUEおよび$E_o$(燃油効率)を算出し、操業実態を定量化している。

このように解析手法をブラックボックス化せず、再現可能な形で共有することは、本研究の透明性を担保するだけでなく、同様の課題を抱える他海域の研究者が本手法を容易に適用することを可能にする。これにより、洋上風力発電と漁業の共存に向けた知見の蓄積に貢献したい。

\begin{figure}[h!]
    \centering
    \includegraphics[width=\textwidth]{images/data-flow.png}
    \caption{本研究におけるデータ処理フローと解析手法の全体像}
    \label{fig:data_flow}
\end{figure}