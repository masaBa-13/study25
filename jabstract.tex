% TODO: 日本語アブストラクトを以下の{}内に記述(以下はダミーテキスト)
\newcommand{\jabstract}{

本研究では、洋上風力発電と沿岸漁業の共存を目的として、操業実態の定量的な可視化に取り組んだ。そして、GNSS航跡データ、水揚げ記録、操業日誌データ、給油履歴を統合し、操業実態を解析する手法を提案した。
北海道松前町のマグロ延縄漁業に提案手法を適用した結果、漁場と促進区域の重複は確認されなかった一方、港から漁場への航路が促進区域を横断していることが明らかになった。また、洋上風車建設により1回の操業における総移動距離が1 nm増えるごとに、年間で約4,330Lの燃油消費量が増加すると試算された。また、7~8月に操業効率がピークとなる季節性や、CPUE・燃油効率などのベースラインを提示することができた。
さらに、操業日誌データがない場合と比較検証した。その結果、操業日誌データを用いない手法では約1.7倍多くのGNSS航跡データ件数を抽出したものの、利用頻度分布図は類似の傾向を示した。
}
