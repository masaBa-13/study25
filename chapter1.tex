\chapter{序論}
\section{背景}

近年、世界規模での環境変動への懸念が急速に高まっている。特に、産業革命以降の温室効果ガス排出量の増加は、地球温暖化と気候変動を引き起こし、異常気象、海面上昇、生態系の変化を通じて人類社会に深刻な影響を与えている。こうした状況を受け、国際社会は気候変動対策を強化する方向へと舵を切るようになった。

その大きな転換点となったのが、2015 年に採択されたパリ協定である。パリ協定は「世界の平均気温上昇を産業革命前と比較して 2 ℃より十分低く抑制し、さらに 1.5 ℃ に抑える努力を追求する」ことを国際的に共有された目標として掲げた枠組みである\cite{Paris_Agreement}。この協定では、すべての締約国に対して温室効果ガス排出削減のための自主的な貢献目標(NDC: Nationally Determined Contribution)を設定し、定期的な報告と更新を課すことで、長期的な気候安定化に向けた取り組みを促進している。
パリ協定の採択を受け、世界各国で再生可能エネルギーの導入が急速に進んでいる。国際再生可能エネルギー機関(IRENA)の統計によれば、2015 年から 2025 年の間に世界の再生可能エネルギー発電容量は、約 1,851 GWから約 4,448 GWへと増加し、特に太陽光発電と風力発電が成長を牽引した\cite{IRENA_2025}。世界的な脱炭素化の潮流の中で、再生可能エネルギーはエネルギー安全保障の観点からも重要度を増している。

日本においてもこの流れは顕著である。2020 年 10 月、日本は「2050 年カーボンニュートラル」を宣言し、2050 年までに温室効果ガス排出量の実質ゼロを目指す方針を明確に打ち出した。この宣言を受けて洋上風力発電の促進を含む多くの政策が整備され、特に再生可能エネルギーの主力電源化が国家戦略として位置づけられるようになった。政府が策定した「第 6 次エネルギー基本計画」では、2030 年までに温室効果ガスを 46 \% 削減し、さらに 50 \% 削減に向けて挑戦を続けるとしている。また、再生可能エネルギーの電源構成比は 36--38 \% を目標としている\cite{Energy_Plan_6th}。

しかし、日本は国土面積が限られているうえ、山岳地域が多く、陸上での大規模風力発電の設置には制約が多い。そのため、近年特に注目されているのが洋上風力発電である。日本の排他的経済水域(EEZ)は世界第 6 位の広さを持ち、風況も比較的安定している地域が多いことから、洋上風力はエネルギー源として高い潜在性がある。政府の導入目標では、2030 年に 10 GW、2040 年に 30--45 GW の導入目標が掲げられており\cite{Green_Growth_Strategy}、日本の再エネ拡大戦略において洋上風力は中核を担う位置付けとなっている。

現状、日本国内では北海道石狩湾新港、秋田県能代・秋田港など複数の地点で着床式洋上風力が稼働しており、2024 年末時点で稼働中の洋上風力発電設備はおよそ 4 地域、総発電容量は約 0.25 GW 程度である\cite{JWPA_Stats_2024}。世界的に見ればまだ導入量は少ないものの、今後 10 年で急速に拡大することが期待されている。

しかし、洋上風力発電の導入には大きな課題も存在する。その最たるものが、地元漁業者や漁協との合意形成の困難さである。秋田県では、洋上風力発電の導入をめぐって漁業者から反発があり、補償金の交渉や操業区域への懸念が解消されず、合意形成が難航した事例が報告されている\cite{akita}。これに限らず、国内外のいくつかの先行研究でも共通して、漁業者は洋上風車建設が漁場環境や操業の安全性、漁獲量にどのような影響を与えるかが明確でないことを懸念していると指摘されている\cite{Aomori,Hooper2015,Reilly2015}。

特に、漁獲量への影響については、科学的な知見が十分に蓄積されていない。例えば、洋上風車の建設や稼働によって、水中音が魚類の行動にどの程度影響するのか、あるいは、洋上風車の設置に伴う海底地形の変化が漁獲量にどのような変化をもたらすのかといった点は、データが不足している。
その結果、多くの議論が漁業者の経験則や個別の証言に依存した定性的なものに留まり、評価が困難になっている。この不透明性こそが、漁業者と事業者の合意形成を阻害し、洋上風力の普及における最大のボトルネックのひとつとなっている。

\section{研究目的}

北海道松前町沖は、日本国内においても風況に恵まれた海域である。2024年7月30日、国は再エネ海域利用法に基づき、同海域を「海洋再生可能エネルギー発電設備整備促進区域(以下、促進区域)」に指定した。これにより、発電事業者の公募・選定および建設に向けたプロセスが今後本格化することとなる。

しかしながら、大規模な洋上風車の建設が既存の漁業活動に与える影響については、未だ定性的な議論に留まることが多い。従来、漁業影響評価は漁業者の経験則やヒアリングに依存する傾向にあり、客観的なデータに基づく検証が十分になされてこなかった。実際、「北海道松前沖における協議会意見とりまとめ」\cite{matsumae_council_2024}においても、発電事業による漁業への支障が生じたか否かを「客観的に認める」ことの重要性が指摘されており、科学的根拠に基づく判断材料の欠如は、事業者と漁業者の合意形成や補償交渉を難航させる大きな要因となり得る。

一方で、横山\cite{Yokoyama2025}は、沖合洋上風力発電の導入に際して漁業関係者が懸念する事項に対応するには、どのような情報が最低限必要であり、その情報を得るためにどの調査手法をどう組み合わせるべきかを明確にすることが重要であると指摘している。また、資源量の変動要因として、洋上風力発電施設の影響と自然変動とを識別できるよう、長期的かつ継続的なデータ蓄積が不可欠であることも述べている\cite{Yokoyama2025}。しかし、全国的な調査スキームの議論が進む一方で、個別の漁業や地域スケールで操業実態を具体的に示した事例は限られており、こうした枠組みを現場レベルで支える実証的なデータの整備が求められている。

本研究の目的は、松前町における主要漁業であるマグロ延縄漁業を対象に、その操業実態を定量的に可視化し、洋上風車建設が漁獲活動に与え得る影響を検証するためのベースラインを構築することである。客観的なデータに基づく現状評価を確立することで、将来的な影響評価を科学的に行い、地域漁業と発電事業の共存に寄与することを目指す。

\section{研究目標}

1.2節の研究目的を達成するため、本研究では以下の具体的な達成目標を設定する。

まず、松前さくら漁業協同組合および地元漁業者の協力のもと、各種データ(GNSS航跡データ、水揚げ記録、給油履歴、および操業日誌データ)を統合した詳細な操業データベースを構築する。その上で、洋上風車建設前の現状として、以下の2点を定量的に明らかにすることを目標とする。

\begin{enumerate}
    \item \textbf{操業海域の空間分布と促進区域との重複状況の解明} \\
    漁業者が利用する漁場および航路を可視化し、2024年7月30日に指定された促進区域(図\ref{fig:sokushin}参照)との地理的な位置関係や重複度合いを定量化する。特に、協議会資料において重要魚種として言及されているマグロの漁期(7月〜翌1月) における空間利用の実態を詳細に特定する。
    
    \item \textbf{漁獲効率および燃油効率の季節変動特性の評価} \\
    季節ごとの海域利用の変化と、それに伴う漁獲効率(CPUE)や燃油効率の推移を算出する。これにより、将来的に風車建設による迂回や漁場喪失が生じた際、それが漁業経営(移動コストや生産性)に与える経済的な影響を試算するための基礎数値を導出する。
\end{enumerate}


\begin{figure}[H]
  \centering
  \includegraphics[width=8cm]{images/sokushin.png} 
  \caption{2024年7月30日に指定された北海道松前沖の促進区域}
  \label{fig:sokushin}
\end{figure}


\section{高度ICTコースにおける本研究の位置付け}

高度ICTコースでは、社会の問題を発見し、ICTを用いてその解決に資する価値ある情報システムの創造を目指す。

本研究は、洋上風力発電の導入に伴う漁業との海域利用調整という、近年の政策的にも社会的にも極めて重要なテーマを対象としている。海域利用の対立は顕在化しており、漁業者の操業実態が十分に可視化されていないことが合意形成を阻害する一つの要因となっている。漁業者がいつ、どこで、どのように操業しているのかという情報は、従来は漁業者自身の経験や口頭での説明やアンケートといった定性的なものに依存しており、第三者が実態把握することは困難であった。
そこで本研究では、漁船位置情報と操業記録を組み合わせ、漁業者の操業実態を定量的に把握する情報システムを構築する。

以上のように、本研究は、社会的課題の解決に ICT を適用する実践的な取り組みであり、情報システム分野の学術的・実務的な両側面に貢献する位置付けにある。
