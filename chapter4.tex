\chapter{分析結果}
本章では、前章で述べた手法に基づき、松前町沖におけるマグロ延縄漁業の操業実態を解析した結果を示す。
本研究では、操業日誌アプリ「マグログ」が導入された2024年漁期のデータを主たる分析対象とし、詳細な空間分布および操業効率を算出した。
また、比較対象としてアプリ導入前の2023年漁期についても、水揚げ記録とGNSS航跡データを統合することで同様の解析を行った。
なお、解析期間は各年ともマグロ漁の主要漁期である7月から翌1月までとした。

\section{空間分布の可視化結果}

\subsection{2024年漁期の操業実態(アプリ導入後)}
操業日誌データとGNSS航跡データを統合し、ゼロキャッチ(漁獲なし)の日を含む全操業日を対象に作成した2024年漁期のヒートマップを図\ref{fig:heatmap2024}に示す。
なお、図中の黄色い枠線は、洋上風力発電の促進区域(または有望な区域)の境界を示している。



\subsection{2023年漁期の操業実態(水揚げ記録ベース)}
比較対象として、アプリ導入前の2023年漁期のヒートマップを図\ref{fig:heatmap2023}に示す。
2023年は操業日誌データが存在しないため、漁協の水揚げ記録に記載のある日付(実際に漁獲があった日)のGNSS航跡データを抽出して解析を行った。

図\ref{fig:heatmap2024}(2024年)と比較すると、主要な高密度海域の位置は概ね一致

\begin{figure}[htbp]
  \centering
  \includegraphics[width=14cm]{images/catch2023map.jpg} % ファイル名は適宜変更
  \caption{2023年漁期の操業密度分布(水揚げ記録・GNSS統合データ)}
  \label{fig:heatmap2023}
\end{figure}

\section{操業効率および燃油効率の分析結果}
% ここ以降は、別途Excel等で計算した数値を埋め込む必要があります。
% 地図からは読み取れないため、前回のテンプレートをベースにしています。

\subsection{単位移動距離あたり漁獲量 (CPUE) の推移}
2023年および2024年漁期における月ごとの単位移動距離あたり漁獲量 (CPUE) の推移を比較した結果を図\ref{fig:cpue_trend}に示す。

\subsection{燃油効率 ($E_o$) の推移}
給油履歴と水揚げ記録に基づいて算出した燃油効率 ($E_o$) の季節変動を図\ref{fig:fuel_efficiency}に示す。

