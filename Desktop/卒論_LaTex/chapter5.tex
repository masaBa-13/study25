\chapter{結果}	% TODO: 章題を記入.題は任意.


\section{空間分布の可視化結果}
\subsection{操業海域の空間的特徴}
ヒートマップ(図)を引用し、「操業密度の高いエリア(赤色)は松前町沖の西側XXkm付近に集中している」といった事実を記述。
\subsection{洋上風力発電促進区域との位置関係}
促進区域の境界線と重ね合わせた結果、「漁場(高密度エリア)は区域外であるが、漁港から漁場へ向かう航跡(航路)は区域内を横断している」という事実を指摘。

\section{操業効率の分析結果}
\subsection{年度別・月別の漁獲量推移}
棒グラフなどを示し、月ごとの漁獲量の変化を説明(例:「7月にピークを迎え、冬場に低下する」)。
\subsection{燃油効率の季節変動特性}
燃油効率 $E_o$ のグラフを示し、「7月〜9月にかけて効率が最も高く、それ以外の時期は低い」というトレンドを数値と共に記述。