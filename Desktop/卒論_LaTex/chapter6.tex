\chapter{考察}	% TODO: 章題を記入.題は任意.

%TODO: 章の内容を記入.以下はサンプル.
この章は最終章である.
第1章と最終章は対比がとれていることが望ましい.
具体的には,「序論」ではじめたのなら「結論」で終わり,
「はじめに」ではじめたのなら「おわりに」で終わる.
「緒言」ではじめたのなら「結言」で終わる.

\section{促進区域における航行安全性と海域利用調整}
航行リスク: 結果5.1.2を受け、「漁場は被っていないから安心」ではなく、「通勤路(航路)が塞がれるリスク」があることを論じる。
具体的影響: 風車を避けるための迂回が発生すれば、燃油コスト増や労働時間増につながる。また、荒天時の避難ルート確保などの安全対策(通航帯の設定など)が必要であると提言。

\section{業効率の季節性に基づく工事時期への提言}
重要時期の特定: 結果5.2.2を受け、7月〜9月は漁業者にとって「少ない燃料でたくさん獲れる(=利益率が高い)」最重要シーズンであると解釈。

工事への提言: この時期に海上工事(騒音、作業船の往来)を行うと、経済的打撃が最大になる恐れがある。したがって、効率の低い時期に工事をずらすなどの配慮が、実質的な共存策として有効であると主張。

\section{洋上風車建設前のベースラインとしての有効性}
比較の基準: 本研究のデータは「風車がない状態(Before)」の貴重な記録である。今後建設された後(After)に同様の調査を行えば、変化を科学的に検証できる(BACIデザインへの適用可能性)とまとめる。