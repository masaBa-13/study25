\chapter{関連研究}

本章では、まず洋上風力発電が漁業活動に及ぼす様々な影響について、国内外の既存研究を概観する。次に、これらの影響を評価するために用いられてきた従来の手法に着目し、その現状と課題を整理する。
最後に、これらの課題を解決する新たなアプローチとして、本研究が提案するGNSS 航跡データと操業記録を統合した解析手法の有効性を示し、本研究が当該分野においてどのような位置付けにあるかを明確にする。

\section{洋上風力発電と漁業の関係性}

洋上風力発電の導入は、海洋空間の利用形態に大きな変化をもたらし、先行利用者である漁業者に対して物理的・社会的な影響を及ぼす。

物理的な影響としては、洋上風車建設時の杭打ち音や稼働音による水中騒音が、魚類の回避行動や生理的ストレスを引き起こす可能性が指摘されている\cite{Boehlert2010, Bailey2014, akamatsu_2018}。一方で、洋上風車の基礎構造物が人工魚礁として機能し、底生生物や魚類の生息密度を高める効果などのポジティブな影響も報告されており\cite{Boehlert2010, Lindeboom2011}、その影響は海域や魚種によって一様ではない。

社会的な側面においては、漁業者と開発事業者との合意形成が最大の課題となっている。Alexander ら (2013) や Hooper ら (2015) は、漁業者が抱く懸念の根本には、漁場の喪失や漁具の破損といった直接的な被害だけでなく、将来の不確実性に対する不安や、開発プロセスにおける信頼関係の欠如があることを指摘している\cite{Hooper2015, Alexander2013}。
日本国内においても、漁獲量への影響が科学的に不透明であることが合意形成を難航させる要因となっており\cite{akita, Aomori}、Shimada ら (2022) が指摘するように、水揚げ記録では検出できない局所的な操業実態の変化を捉えることが求められている\cite{Shimada2022}。

\section{既存の調査手法とその限界}

洋上風力発電の漁業影響を評価するために、従来は主に科学的調査や水揚げ記録による分析の手法が用いられてきた。

科学的調査には、調査船を用いたトロール網調査などの生物学的調査が含まれる。これらは客観的な生物データを取得できる反面、広域かつ連続的な観測を行うには多額のコストと継続的な予算確保が必要となるため、調査頻度が限られ、時空間的な解像度が不足するという課題がある\cite{Zhu2021}。

一方、漁協の水揚げ記録や従来の操業日誌を用いた分析は、長期間の漁獲量の変化を把握する上では有効である。しかし、これらのデータには正確な位置情報が記録されていないため、漁業者がいつ、どこで、どのように操業していたのかを詳細に把握することは難しい。Shimada ら(2022)は、水揚げ記録のみでは局所的な操業の変化や空間的な利用実態を十分に評価することができず、洋上風力発電の影響検出においても限界があることを指摘している\cite{Shimada2022}。

さらに、Allen-Jacobson ら(2023)は、位置情報を持たない水揚げ記録に基づく分析では、洋上風車と漁業活動との空間的な関係を高精度には把握できず、その影響を過大または過小に評価してしまう可能性があることを示している\cite{AllenJacobson2023}。

\section{本研究のアプローチと位置付け}

上述した既存手法の限界に対し、近年では ICT 技術を活用したスマート水産業の取り組みが進められている。和田ら (2011) は、小型漁船の GNSS 位置情報と操業日誌を統合して解析することで、従来の手法では把握が困難であった資源の微細な空間分布や資源量を、高精度に可視化・推定できることを実証している\cite{Wada2011}。

本研究は、この和田ら (2011) の高精細な可視化手法を、洋上風力発電の影響評価という新たな課題領域に応用する点に特徴がある。Allen-Jacobson ら (2023) が指摘したような粗い位置情報に起因する評価の過大・過小評価を克服するため、本研究では 30 秒間隔の GNSS 航跡データと操業日誌などを統合し、より詳細に空間利用と操業効率を定量化する。
これにより、定性的な議論に留まりがちであった漁業影響評価を、客観的なデータに基づく定量的なプロセスへと転換することを目指す。