\chapter{関連研究}

本章では、洋上風力発電が海洋環境や地域社会に与える影響に関する既存研究、およびその調査手法について述べる。洋上風力発電の導入は世界的に加速しているが、漁業との共存は共通の課題となっている。

先行研究は主に、1. 海洋生態系への物理的・生物学的影響、2. 従来の漁業資源調査および影響評価手法、3. 漁業者や地域社会の受容性、4. 統計的因果推論を用いた経済的影響評価、に大別される。本節ではこれらの知見と手法の限界を整理し、本研究の位置付けを明確にする。

\section{環境および生態系への影響}

洋上風力発電設備の建設・稼働が海洋生態系に与える影響については、洋上風力発電が盛んな欧州を中心に知見が蓄積されている。Boehlert and Gill (2010) は、海洋再生可能エネルギー開発に伴う騒音、電磁界、生息地の改変といった環境ストレス要因を包括的に整理し、これらが海洋生物に及ぼす影響は、施設の規模や立地環境によって大きく異なると指摘している\cite{Boehlert2010}。この指摘は、洋上風力発電の影響が一様ではなく、建設予定海域ごとの詳細な実態把握に基づいた影響評価が不可欠であることを示唆している。

また、個別の影響要因に関する研究として、Bailey et al. (2014) が、風車の建設段階における杭打ち音が海洋哺乳類や魚類に回避行動を引き起こす可能性を指摘している。彼らは、建設工事が一時的に生物の分布を変える可能性がある一方で、長期的な個体数への影響を検出するには、長期間にわたる継続的なモニタリングが必要であると結論づけている\cite{Bailey2014}。

国内においても、環境省の技術講習会などで、水中騒音がマダイ等の魚類や海産哺乳類に与える生理的・行動的影響(聴覚閾値や威嚇反応など)を定量的に評価する手法や、パッシブ音響モニタリング等の新たな調査技術の導入に関する議論が進められている\cite{akamatsu_2018}。

また、Lindeboom et al. (2011) は、オランダ沿岸に位置する洋上風力発電所(エグモント・アーン・ゼー洋上風力発電所、OWEZ)における短期的な生態学的影響を包括的に調査し、風車基礎や洗掘防止用の岩石が新たな生息基盤となることで、底生生物やカニ類、魚類の生物多様性および個体数が増加したことを報告している。彼らは、風車施設が実質的な保護区や人工魚礁として機能し、特定の生物群集に対してプラスの効果をもたらす可能性を示唆している。\cite{Lindeboom2011}。この研究は、洋上風力発電が生態系に与える影響が多面的であり、ネガティブな影響だけでなく、ポジティブな効果も存在し得ることを示している。

\section{従来の漁業影響調査および資源調査手法}

洋上風力発電の建設前後における漁業資源量の変化や、漁業への影響を評価するために、従来から用いられている標準的な手法が存在する。これらは主に調査船などを用いる「科学的調査」と、漁業活動から得られるデータを利用する「漁業情報に基づく調査」に分類される。

\subsection{調査船による調査}
資源量推定において最も標準的に用いられるのが、調査船を用いた調査である。特に底魚類を対象としたトロール網調査は、定められた地点において科学的調査船が試験操業を行うものであり、対象海域の生物量や種組成を客観的に把握する手法として確立されている。環境アセスメントの文脈においても、国内の指針に基づき、建設予定海域における事前のトロール調査や流し網調査等による魚類の分布状況把握が求められている\cite{meti_yojo_2023}。

しかし、これらの手法は調査船の運航コストが高額であるため、調査頻度が季節に1回や年に数回といった低頻度に限られることが多い。また、トロール調査は点や線の情報であるため、洋上風車のような局所的な構造物が魚群の微細な移動経路や分布に与える影響を捉えるには、時空間的な解像度が不足しているという課題がある。

\subsection{操業活動による調査}
もう一つの手法は、漁業者の操業活動から得られるデータを用いる調査である。これには、漁協の水揚げ統計や、漁業者が記録する操業日誌が含まれる。これらのデータは、単位努力量あたりの漁獲量である CPUE を算出することで、資源量の変動トレンドを把握するために用いられる。

従来の紙媒体による操業日誌は、位置情報が漁区単位や漁場名といった粗い粒度でしか記録されないことが多く、Shimada et al. (2022) が指摘するように、洋上風車建設による局所的な漁場の喪失やミクロな影響を検出するには限界があった\cite{Shimada2022}。

しかし近年では、ICT技術の活用により、この解像度の課題を克服する試みもなされている。例えば、和田ら (2013) は、小型漁船にGNSSロガーやタブレット端末を導入することで、詳細な操業位置と漁獲情報を紐付け、高精細な資源量分布の可視化が可能であることを実証している\cite{Wada2013}。このアプローチは、漁獲依存型データであっても、適切なデジタル化を行うことで、洋上風力発電の影響評価に求められる高い時空間解像度を達成できる可能性を示唆している。

\section{漁業者の認識と社会的受容性}

洋上風力発電の導入に対する漁業者の態度や懸念に関しては、社会科学的なアプローチによる研究が行われている。Alexander et al. (2013) は、スコットランドの漁業者を対象とした調査において、開発事業者と漁業コミュニティの間の信頼の欠如が紛争の主要因であると指摘している\cite{Alexander2013}。彼らの研究によれば、漁業者は自分たちの知識やデータが計画に反映されていないと感じており、この手続き的公正の欠如が、プロジェクトへの強い反発を生んでいる。

Hooper et al. (2015) は、英国における洋上風力発電所とカニ・エビ漁業の共存の可能性について、漁業者と開発者双方へのインタビューを実施した。その結果、漁業者は風車建設による操業エリアの実質的な喪失や漁具の破損リスク、航行安全性の低下を具体的な懸念として挙げており、物理的な共存が可能であっても、心理的・実務的な障壁が高いことを明らかにしている\cite{Hooper2015}。

これらの先行研究は、漁業者の懸念が単なる感情的なものではなく、操業実態や将来の不確実性に根ざしたものであることを示している。しかし、これらの多くは定性的な調査に留まっており、懸念の妥当性を検証するための定量的なデータ分析は十分に行われていない。

\section{経済的影響の定量的評価と因果推論}

再生可能エネルギー施設の導入が地域経済に与える影響を、統計データを用いて定量的に評価する試みも進められている。Jensen et al. (2018) は、デンマークにおける陸上および洋上風力発電所の建設が周辺の地価に与える影響を分析したが、統計的に有意な影響は見られなかったと報告している\cite{Jensen2018}。また、Ando (2015) は、日本における原子力発電所の立地が地域経済に与える影響を Synthetic Control Method、いわゆる SCM を用いて分析し、因果効果推定の有効性を示した\cite{Ando2015}。

Shimada et al. (2022) は、この SCM の手法を日本の洋上風力発電と漁業生産の関係に応用した画期的な研究である。彼らは、銚子・北九州・五島の 3 地域の海面漁業生産統計を分析し、風車建設が地域の漁獲量全体に対して統計的に有意な負の影響を与えていないことを明らかにした\cite{Shimada2022}。
しかし、Shimada et al. (2022) も指摘しているように、自治体単位のマクロな統計データでは、個々の漁業者が直面する局所的な漁場の変化や、操業効率の低下といったミクロな影響を捉えきれないという課題が残されている。

\section{本研究の位置付け}

以上の関連研究を踏まえると、現状の課題として以下の 3 点が浮き彫りとなる。

第一に、洋上風力発電の環境影響評価に関する制度報告では、魚類・底生生物・音響・流況など多様な評価項目を含めるべきことが指摘されている一方で、既存の海洋資源調査が必ずしも漁業への影響を正確に測れていない可能性があるとして、知見不足だというという留保が示されている。\cite{moe_offshore_2022}。

第二に、Shimada et al. (2022) が行ったようなマクロな統計解析では、Hooper et al. (2015) が指摘した局所的な操業エリアの喪失や航行リスクを直接的に検証できない点である\cite{Shimada2022, Hooper2015}。

本研究は、GNSS ロガーによる操業位置情報と、操業記録を統合することで、これらの課題に応えるものである。低頻度な科学調査や粗い統計データといった従来の手法では捉えきれなかった、漁業者がどこで、どのように操業しているかという具体的なプロセスを可視化し、将来の風車建設に向けた精緻なベースラインを構築することを目的とする。