% TODO: 論文題目等の情報を以下に記入

\newcommand{\jtitle}{沿岸漁業と洋上風力発電の共存に向けた海域利用の可視化}
\newcommand{\etitle}{Visualizing an utilization of sea area for the coexistence of coastal fisheries and offshore wind power}
%タイトル:APAスタイルとします.特に大文字化の指示がない限り、単語はすべて小文字で表記します。例えば、文頭の単語は大文字で表記しますが、文頭の人物名が小文字で始まる場合は除きます。
%参考:Capitalization https://apastyle.apa.org/style-grammar-guidelines/capitalization
%参考:Title Case Converter – A Smart Title Capitalization Tool https://titlecaseconverter.com/
\newcommand{\jauthor}{祐川雅治}
\newcommand{\eauthor}{Masaharu Sukekawa}

\newcommand{\jadvisor}{和田雅昭}
\newcommand{\eadvisor}{Masaaki Wada}
%教員名の表記は以下のサイトの表記に従うこと
%教員 – 公立はこだて未来大学 -Future University Hakodate- https://www.fun.ac.jp/faculty
\newcommand{\jdate}{2026年1月27日}  % 論文提出日   (日)
\newcommand{\edate}{January 27th, 2026}  % 論文提出年月 (英)
\newcommand{\jkeywords}{GIS, 可視化, 洋上風力発電, 漁業, 松前町} % キーワード(日)
\newcommand{\ekeywords}{GIS, Visualization, Offshore wind power, Fisheries, Matsumae}   % キーワード(英)
\newcommand{\eshorttitle}{Visualizing an utilization of sea area}    % 短縮英題題名(おおよそ8 words以内)
\newcommand{\jdepartment}{情報アーキテクチャ学科}    % 学科名(日)
%\newcommand{\jdepartment}{複雑系知能学科}    % 学科名(日)
\newcommand{\jcourse}{高度ICTコース}    % コース名(日)
%\newcommand{\jcourse}{高度ICTコース}    % コース名(日)
%\newcommand{\jcourse}{情報デザインコース}    % コース名(日)
%\newcommand{\jcourse}{複雑系コース}    % コース名(日)
%\newcommand{\jcourse}{知能システムコース}    % コース名(日)
\newcommand{\studentID}{1022134}    % 学籍番号
\newcommand{\edepartment}{Department of Media Architecture}    % 学科名(英)
%\newcommand{\edepartment}{Department of Complex and Intelligent Systems}    % 学科名(英)
\newcommand{\ecourse}{Advanced ICT Course}    % コース名(英)
%\newcommand{\ecourse}{Advanced ICT Course}    % コース名(英)
%\newcommand{\ecourse}{Information Design Course}    % コース名(英)
%\newcommand{\ecourse}{Complex Systems Course}    % コース名(英)
%\newcommand{\ecourse}{Intelligent Systems Course}    % コース名(英)
