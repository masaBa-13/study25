\chapter{研究対象と使用データ}
本章では、本研究の実証フィールドである北海道松前町の地域特性および対象漁業の概要を述べるとともに、解析に用いたデータセットの仕様、およびデータの統合・前処理の手順について詳述する。特に、複数のデータソース(GNSS、水揚げ記録、操業日誌、燃油履歴)を統合し、マグロ延縄漁業の操業実態を正確に抽出するための独自の手法について重点的に説明する。
\section{対象フィールドと漁業概要}本研究の対象フィールドは、北海道南端に位置する松前町沖である。この海域は対馬海流が流入する津軽海峡に面しており、クロマグロ(以下、マグロ)をはじめとする豊かな水産資源に恵まれた好漁場である。対象とする漁業は、松前さくら漁業協同組合に所属するマグロ延縄漁業である。本研究では、同漁協所属の漁船29隻を分析対象とした。
\section{使用データセット}本研究では、漁船の行動、漁獲実態、および生産コスト(燃油)を定量化するために、以下の4種類のデータを統合して使用した。
\subsection{GNSS航跡データ}対象漁船に搭載されたGNSSロガーから取得された航跡データである。\begin{itemize}\item \textbf{データ形式}: CSV形式\item \textbf{サンプリング間隔}: 30秒\item \textbf{収録項目}: 漁船ID、日時(JST)、緯度、経度、対地速度(knot)、進行方向\end{itemize}本データは、漁船の高精細な移動経路を把握するための基礎情報として用いる。\subsection{漁獲・水揚げ記録}松前さくら漁協の販売管理システムより抽出された、2022年から2025年11月までのマグロの水揚げデータである。\begin{itemize}\item \textbf{対象期間}: 2022年〜2025年11月\item \textbf{解像度}: 1個体ごとのレコード(1匹単位で管理)\item \textbf{対象魚種}: クロマグロのみ\item \textbf{収録項目}: 水揚げ日、漁船ID、重量(kg)、単価など\end{itemize}本データは「実際に水揚げされた(換金された)正確な重量」を示す正解データとして機能するが、水揚げが無かった日(操業したが釣れなかった日)の情報が含まれないという制約がある。\subsection{操業日誌データ(マグログ)}漁業者がスマートフォン上の操業日誌アプリ(通称:マグログ)に入力した記録データである。本研究において、このデータは以下の2つの重要な役割を担う。\begin{enumerate}\item \textbf{マグロ操業日の特定(ゼロキャッチの把握)}:松前町の漁業者は、マグロ漁期であっても海況や時期により他の魚種(イカ、根魚など)を狙う場合がある。GNSSデータ単体では「何の魚を獲りに行ったか」を判別することは困難である。また、前述の「水揚げ記録」だけでは、マグロ狙いで出漁したが漁獲が無かった日(ゼロキャッチ)を把握できない。操業日誌データには、漁獲がゼロであった場合でも操業記録が残るため、これをGNSSデータと照合することで、当該航海が「マグロ延縄操業」であったことを正確に特定することが可能となる。\item \textbf{サイズ別漁獲尾数の把握}:本データには、漁獲したマグロのサイズ区分ごとの本数が記録されている。サイズ区分は以下の通りである。\begin{itemize}\item $\sim$25kg, $\sim$30kg, $\sim$50kg, $\sim$75kg, $\sim$100kg, 100kg$\sim$\end{itemize}\end{enumerate}なお、操業日誌のサイズ区分および入力値は漁業者の目測に基づくものであるため、漁協の公式な水揚げ記録と比較して重量のズレが生じる可能性がある。この信頼性を検証するため、2024年のデータを用いて両者の総重量の差異を計測したところ、ズレ(誤差)は約6.69\%に留まることが確認された(日誌データの重量は各サイズ区分の中央値を用いて算出)。この結果から、本操業日誌データは分析に用いる上で十分な精度を有していると判断した。\subsection{燃油給油履歴}漁業者が漁協を通じて精算した給油代金の履歴データである。\begin{itemize}\item \textbf{収録項目}: 精算日、漁船ID、油種(A重油または軽油)、給油量(L)\item \textbf{データの性質}: 本データは毎回の航海ごとの消費量を計測したものではなく、1週間から2週間ごとの給油および精算のタイミングで記録されたものである。したがって、本研究では日次単位ではなく、月次単位でデータを集約することで燃油消費量を算出した。\end{itemize}\section{データ統合と前処理手法}収集されたデータは形式や記録頻度が異なるため、Pythonを用いて以下の手順で統合およびクリーニングを行った。\subsection{データのクリーニングと正規化}
まず、生データに含まれる欠損値や物理的にあり得ない異常値を除外した。また、データソースによって漁船名の表記揺れが存在したため、船名対応表(マスタデータ)を作成し、すべてのデータセットにおいて漁船ID(\texttt{vessel\_id})の正規化を行った。\subsection{マグロ操業の特定とフィルタリング}本研究の分析対象となる「マグロ延縄操業」の航跡のみを抽出するため、以下の多段階のフィルタリング処理を実装した。\begin{enumerate}\item \textbf{異常値の除外}:GNSSデータにおいて、対地速度が25ノットを超えるレコード、および計測点間の移動距離が1km以上となるレコード(位置飛び)を、測定エラーとして除外した。\item \textbf{操業日誌との照合(Intent Matching)}:GNSSデータと操業日誌データを「漁船ID」および「日付」をキーとして結合(Inner Join)した。これにより、漁業者が「マグロ漁を行った」と記録している日の航跡のみを抽出した。\item \textbf{航海(Trip)の分離}:GNSSデータ上で記録間隔が1時間以上空いている場合、別の航海(または操業)が行われたと判断し、セッションを分割した。\item \textbf{移動距離によるフィルタリング}:抽出された航跡データの中から、港内での作業やごく近距離での他漁業(採介藻漁業等)を除外するため、1回の航海の総移動距離を用いたフィルタリングを行った。松前町の漁業者へのヒアリング調査により、マグロ延縄漁の平均的な移動距離は24〜30マイル(約44〜55km)程度であることが示された。この知見に基づき、本研究では保守的な閾値として\textbf{6マイル(約11km)}を設定し、これ未満の航海データを分析対象から除外した。\end{enumerate}\subsection{データの集計(Aggregation)}以上の処理を経て抽出されたデータを基に、月別および漁船別のクロス集計を行った。燃油データが日次で存在しないことを考慮し、漁獲量(水揚げ記録に基づく重量)と燃油消費量をそれぞれ月単位で合計し、月ごとの燃油効率 $E_o$(kg/L)を算出した。これらの集計結果はCSV形式で保存し、後の可視化プロセスに使用した。