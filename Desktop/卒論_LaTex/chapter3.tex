\chapter{研究対象と使用データ}
本章では、本研究の対象フィールドである北海道松前町の地域特性および対象漁業の概要を述べるとともに、解析に用いたデータセットの仕様、およびデータの統合・前処理の手順について記述する。特に、複数のデータソース(GNSS航跡データ、水揚げ記録、操業日誌データ、給油履歴)を統合し、マグロ延縄漁業の操業実態を抽出するための手法について説明する。

\section{対象フィールドと漁業概要}本研究の対象フィールドは、北海道南端に位置する松前町沖である。この海域は対馬海流が流入する津軽海峡に面しており、マグロをはじめとする豊かな水産資源に恵まれた好漁場である。対象とする漁業は、松前さくら漁業協同組合に加入する漁業者によるマグロ延縄漁業である。本研究では、これらの漁業者が使用する漁船29隻を分析対象とした。

\section{使用データセット}本研究では、漁船の行動、漁獲実態、および操業効率を定量化するために、表3.1に示す4種類のデータを統合して使用した。

\begin{table}[h!]
    \centering
    \caption{使用データセットの概要と収録期間}
    \label{tab:data_period}
    \begin{tabular}{|l|l|} \hline
        データ名称 & 収録期間 \\ \hline \hline
        GNSS 航跡データ & 2023年7月 $\sim$ 2025年11月末 \\
        水揚げ記録 & 2022年7月 $\sim$ 2025年11月末 \\
        操業日誌データ & 2024年7月 $\sim$ 2025年11月末 \\ 
        給油履歴 & 2022年12月末 $\sim$ 2025年6月末 \\
        \hline
    \end{tabular}
\end{table}

各データセットの詳細について以下に説明する。
\subsection{GNSS航跡データ}対象漁船に搭載されたGNSSロガーから取得された航跡データである。
\begin{itemize}
    \item \textbf{サンプリング間隔}: 30秒
    \item \textbf{収録項目}: 漁船ID(PK)、日時、緯度、経度、速度、進行方向など
\end{itemize}

\subsection{水揚げ記録}
松前さくら漁協で管理しているマグロの水揚げデータである。
\begin{itemize}
    \item \textbf{対象魚種}: クロマグロのみ
    \item \textbf{収録項目}: 漁業者名(PK)、水揚げ日、マグロ重量など
\end{itemize}
本データは実際に水揚げされた正確な重量を示すデータとして機能するが、操業した上で漁獲できなかった日の情報が含まれないという制約がある。

\subsection{操業日誌データ}
漁業者がスマートフォン上の操業日誌アプリに入力したデータである。利用しているアプリは「マグログ」(2024年度に公立はこだて未来大学プロジェクト学習「スマート道南」チーム開発)\cite{maglog}であり、漁業者が操業終了後に日誌情報を入力し、クラウド上に保存する仕組みとなっている。
\begin{itemize}
    \item \textbf{収録項目}: 漁船名(PK)、操業日、海区番号、サイズ区分ごとの漁獲数など
    \item \textbf{サイズ区分}: $\sim$30kg, $\sim$50kg, $\sim$75kg, $\sim$100kg, 100kg$\sim$
\end{itemize}

本研究においてこのデータは、GNSS航跡データがマグロの漁獲を行っている時のものかどうかを判断するために使用する。松前町のマグロ漁業者は、マグロ漁期であっても他の魚種を漁獲している場合がある。GNSS航跡データ単体では何を漁獲していたかを判別することは困難である。また、前述の水揚げ記録だけでは、マグロ狙いで出漁したが漁獲が無かった日を把握できない。操業日誌データには、漁獲が無かった場合でも漁業者が操業記録を残しているため、これをGNSS航跡データと照合することで、当該操業がマグロ延縄漁であったことを特定することが可能となる。

なお、操業日誌のサイズ区分および入力値は漁業者の目測に基づくものであるため、漁協で記録されている水揚げ記録と比較して重量のズレが生じる可能性がある。この信頼性を検証するため、2024 年のデータを用いて両者の総重量の差異を計測したところ、誤差は約 6.69 \% に留まることが確認された(操業日誌データの重量は各サイズ区分の中央値を用いて算出)。この結果から、本操業日誌データは分析に用いる上で十分な精度を有していると判断した。

\subsection{給油履歴}漁業者が漁協を通じて精算した給油代金の履歴データである。
\begin{itemize}
    \item \textbf{収録項目}:漁船名(PK)、取引日、油種(A重油または軽油)、給油量
    \item \textbf{データの性質}: 本データは毎回の航海ごとの消費量を計測したものではなく、約8週間ごとの給油および精算のタイミングで記録されたものである。
\end{itemize}

\section{データ統合と前処理手法}収集されたデータは形式や記録頻度が異なるため、Pythonを用いて以下の手順で統合およびクリーニングを行った。

\subsection{データのクリーニングと正規化}
まず、GNSS航跡データに含まれる欠損値や異常値を除外した。また、データソースによって漁船名の表記揺れが存在したため、船名対応表を作成し、すべてのデータセットにおいて漁船IDの正規化を行った。

\subsection{マグロ操業の特定とフィルタリング}
本研究の分析対象となるマグロ延縄操業の航跡のみを抽出するため、以下のフィルタリング処理を実装した。
\begin{enumerate}
    \item \textbf{異常値の除外}:GNSS航跡データにおいて、速度が25ノットを超えるレコード、およびセンサの異常と思われる、計測点間の移動距離が1マイル以上となるレコードを除外した。
    \item \textbf{操業日誌との照合}:GNSS航跡データと操業日誌データを漁船IDおよび日付をキーとして結合した。これにより、マグロ漁を行った日の航跡のみを抽出した。
    \item \textbf{操業の切り分け}:GNSS航跡データの先頭から操業IDを振り、データ上で記録間隔が 1時間以上空いている場合、別の操業が行われたと判断し、操業IDを更新した。
    \item \textbf{移動距離によるフィルタリング}:同一の操業IDを持つGNSS航跡データの中から、港内での作業や漁港間の移動などを除外するため、1回の航海の総移動距離を用いたフィルタリングを行った。松前町の漁業者へのヒアリング調査により、マグロ延縄漁の平均的な移動距離は16〜27マイル(約30〜50km)程度であることが示された。この知見に基づき、本研究では保守的な閾値として6マイル(約11km)を設定し、閾値未満の航海データを分析対象から除外した。
\end{enumerate}