\chapter{分析手法}	% TODO: 章題を記入.題は任意.
ここでは**「どうやって分析したか」**(Methods)を書く。ここを読めば他の人も同じ実験ができるように書く

\section{GISを用いた空間利用の可視化}
何をしたか: 対象海域を「直径1,000m(または面積1km²など正確に)」の六角形グリッドで覆った。
なぜ六角形か: 正方形グリッドに比べて中心間の距離が等しく、生物の行動圏や連続的な移動データの表現に適しているため(という一般論を少し入れると良い)。使用ソフト(QGIS Ver. X.X)も記載。
\subsection{六角形グリッドによる空間分割}
各グリッド内に含まれるGNSSポイント数をカウントする「ポイント・イン・ポリゴン」処理を行ったことを記述。
\subsection{Jenks自然分類法を用いたヒートマップ作成}
分類手法: 単なる等間隔ではなく「Jenksの自然分類法(Natural Breaks)」を採用した理由(データの偏りを考慮し、クラス間の分散を最大化するため)を説明。
Jenks自然分類の公式を載せると良い。

可視化: 操業密度を5段階の色分けで表現し、ヒートマップ化した手順


\section{操業効率および燃油効率の定義}

\subsection{球面距離(Haversineの公式)による移動距離算出}
GNSSの緯度経度から移動距離を求めるために使用した数式(中間報告書にある $d = 2R \arcsin...$ の式)を提示し、各変数の定義($R$=地球半径など)を説明。Pythonで実装したことも記載。

\subsection{燃油効率指標 $E_o$ の定義と算出モデル}
本研究で定義した効率の計算式 $E_o = W / O$ ($W$:総漁獲重量、$O$:燃油使用量)を提示。なぜこの指標を用いたか(単位燃油あたりの生産性を測るため)を簡潔に。
