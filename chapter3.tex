\chapter{提案手法}
本章では、本研究の対象フィールドである北海道松前町沿岸の海域特性および対象漁業の概要を述べるとともに、解析に用いたデータセットの仕様、およびデータの統合・前処理の手順について記述する。特に、複数のデータソース(GNSS航跡データ、水揚げ記録、操業日誌データ、給油履歴)を統合し、マグロ延縄漁業の操業実態を抽出するための手法について説明する。

\section{対象フィールドと漁業概要}
本研究の対象フィールドは、北海道南端に位置する松前町沿岸である。この海域は対馬海流が流入する津軽海峡に面しており、マグロをはじめとする豊かな水産資源に恵まれた好漁場である。対象とする漁業は、松前さくら漁業協同組合に加入する漁業者によるマグロ延縄漁業である。同漁業協同組合は、令和5年現在で正組合員217名が所属しており、各種網漁業、延縄漁業、釣り漁業、採介藻漁業および養殖漁業を複合的に営んでいる。その中でマグロ漁業の水揚げ金額は、全体の7.7\%を占めている。
本研究では、これらの漁業者が使用する漁船29隻を分析対象とした。

\section{使用データセット}本研究では、漁船の動静、漁獲実態、および操業効率を定量化するために、表3.1に示す4種類のデータを統合して使用した。

\begin{table}[h!]
    \centering
    \caption{使用データセットの名称と収録期間}
    \label{tab:data_period}
    \begin{tabular}{|l|l|} \hline
        データ名称 & 収録期間 \\ \hline \hline
        GNSS 航跡データ & 2023年7月 $\sim$ 2025年11月末 \\
        水揚げ記録 & 2022年7月 $\sim$ 2025年11月末 \\
        操業日誌データ & 2024年7月 $\sim$ 2025年11月末 \\ 
        給油履歴 & 2022年12月末 $\sim$ 2025年6月末 \\
        \hline
    \end{tabular}

\end{table}
\begin{figure}[h!]
    \centering
    \includegraphics[width=12cm]{images/period.png}
    \caption{使用データセットの収録期間}
    \label{fig:data_period}
\end{figure}

各データセットの詳細について以下に説明する。
\subsection{GNSS航跡データ}対象漁船に搭載されたVMS(Vessel Monitoring System)から取得された航跡データである。
\begin{itemize}
    \item \textbf{サンプリング間隔}: 30秒
    \item \textbf{収録項目}: 漁船ID(PK)、日時、緯度、経度、速度、進行方向など
\end{itemize}

\subsection{水揚げ記録}
松前さくら漁協で管理しているマグロの水揚げデータである。
\begin{itemize}
    \item \textbf{対象魚種}: クロマグロ
    \item \textbf{収録項目}: 漁業者名(PK)、水揚げ日、マグロ重量など
\end{itemize}
本データは実際に水揚げされた正確な重量を示すデータとして機能するが、操業した上で漁獲できなかった日の情報が含まれないという制約がある。

\subsection{操業日誌データ}
漁業者がスマートフォン上の操業日誌アプリに入力したデータである。利用しているアプリは「マグログ」(2024年度に公立はこだて未来大学プロジェクト学習「スマート道南」チーム開発)\cite{maglog}であり、漁業者が操業終了後に日誌情報を入力し、クラウド上に保存する仕組みとなっている。
\begin{itemize}
    \item \textbf{収録項目}: 漁船名(PK)、操業日、海区番号、サイズ区分ごとの漁獲数など
    \item \textbf{サイズ区分}: $\sim$30kg, $\sim$50kg, $\sim$75kg, $\sim$100kg, 100kg$\sim$
\end{itemize}

本研究においてこのデータは、GNSS航跡データがマグロの漁獲を行っている時のものかどうかを判断するために使用する。松前町のマグロ漁業者は、マグロ漁期であっても他の魚種を漁獲している場合がある。GNSS航跡データ単体では何を漁獲していたかを判別することは困難である。また、3.2.2の水揚げ記録だけでは、マグロ狙いで出漁したが漁獲が無かった日を把握できない。操業日誌データには、漁獲が無かった場合でも漁業者が操業記録を残しているため、これをGNSS航跡データと照合することで、当該操業がマグロ延縄漁であったことを特定することが可能となる。

なお、操業日誌のサイズ区分および入力値は漁業者の目測に基づくものであるため、漁協で記録されている水揚げ記録と比較して重量のズレが生じる可能性がある。この信頼性を検証するため、2024 年のデータを用いて両者の総重量の差異を計測したところ、誤差は約 6.69 \% に留まることが確認された(操業日誌データの重量は各サイズ区分の中央値を用いて算出)。この結果から、本操業日誌データは分析に用いる上で十分な精度を有していると判断した。
\begin{figure}[h!]
    \centering
    \includegraphics[width=12cm]{images/gosa.png}
    \caption{水揚げ記録と操業日誌入力値の重量誤差}
    \label{fig:data_period}
\end{figure}

\subsection{給油履歴}松前さくら漁協で管理している給油代金の履歴である。
\begin{itemize}
    \item \textbf{収録項目}:漁船名(PK)、給油日、油種(A重油または軽油)、給油量
    \item \textbf{データの性質}: 本データは毎回の操業ごとの給油を計測したものではなく、給油のタイミングで記録されたものである。
\end{itemize}

\section{データ統合と前処理手法}収集されたデータは形式や記録頻度が異なるため、Pythonを用いて以下の手順で統合およびフィルタリングを行った。

\subsection{マグロ操業の特定とフィルタリング}
本研究の分析対象となるマグロ延縄操業の航跡のみを抽出するため、以下のフィルタリング処理を実装した。
\begin{enumerate}
    \item \textbf{異常値の除外}:GNSS航跡データにおいて、速度が25ノットを超えるレコード、およびセンサの異常と思われる、計測点間の移動距離が1マイル以上となるレコードを除外した。
    \item \textbf{操業日誌との照合}:GNSS航跡データと操業日誌データを漁船IDおよび日付をキーとして結合した。これにより、マグロ漁を行った日の航跡のみを抽出した。
    \item \textbf{操業の切り分け}:GNSS航跡データの先頭から操業IDを振り、データ上で記録間隔が 1時間以上空いている場合、別の操業が行われたと判断し、操業IDを更新した。
    \item \textbf{移動距離によるフィルタリング}:同一の操業IDを持つGNSS航跡データの中から、港内での作業や漁港間の移動などを除外するため、1回の総移動距離を用いたフィルタリングを行った。松前町の漁業者へのヒアリング調査により、マグロ延縄漁の平均的な移動距離は16〜27マイル(約30〜50km)程度であることが示された。この知見に基づき、本研究では保守的な閾値として6マイル(約11km)を設定し、閾値未満の航海データを分析対象から除外した。
\end{enumerate}

\section{分析手法と評価指標}
本節では、3.3.1の処理を経て構築されたデータセットを用い、マグロ延縄漁業の操業実態を定量的に明らかにするための分析手法と評価指標について述べる。
なお、本研究における空間解析には、オープンソースの地理情報システムである QGIS(Ver. 3.42.1)を使用した。
\subsection{GISを用いた空間利用の可視化}
% 六角形グリッド、Jenks法の説明
洋上風車の建設予定海域(促進区域)と漁業活動の空間的な重複状況を評価するため、以下の手順で漁船の操業密度分布を可視化した。

\begin{enumerate}
    \item \textbf{六角形グリッドによる空間分割} \\
    分析対象海域を内接円の直径が1 km の六角形グリッドで分割し、空間データの集計単位とした。六角形グリッドは、正方形グリッドと比較して隣接するセルの中心間距離が等しいため、移動体の経路や分布密度を表現する際の方向依存性(異方性)が低く、空間的なバイアスを軽減できる利点がある。 QGIS の「六角形グリッドを作成(Create Hexagonal Grid)」ツールを用いてグリッドを生成した。

    \item \textbf{GNSS 航跡データの集計} \\
    CSV 形式でインポートした GNSS 航跡データの各計測点が、作成した六角形グリッドのいずれに属するかを特定し、各グリッドに含まれる計測点数(ポイント数)を算出した。この集計処理には、QGIS の解析ツールである「ポリゴン内のポイント数を数える(Count Points in Polygon)」を使用した。

    \item \textbf{Jenks 自然分類法を用いたヒートマップ作成} \\
    集計された各グリッドのポイント数を「操業密度」と定義し、その空間分布をヒートマップとして可視化した。密度の階級区分には、Jenks の自然分類法(Jenks Natural Breaks Classification)を採用した。本手法は、データのヒストグラム(度数分布)における「自然な切れ目(Natural Breaks)」を統計的に検出し、クラス内部の差異を最小化しつつ、クラス間の差異を最大化するように境界値を決定する。これにより、データ値が近似しているグリッド同士を適切にグルーピングできる。本アルゴリズムは、全データの平均からの偏差平方和($SDAM$)に対する、各クラスの平均からの偏差平方和の総和($SDCM$)の比率を最小化するように、クラス境界を決定するアルゴリズムである \cite{Jenks1967}。具体的には、以下の式で表される「分散の適合度(Goodness of Variance Fit: GVF)」を最大化することを目的とする。
    \begin{equation}
    GVF = 1 - \frac{SDCM}{SDAM}
    \end{equation}
    ここで、$SDAM$(Sum of Squared Deviations from Array Mean)および $SDCM$(Sum of Squared Deviations from Class Means)は次式で定義される。
    \begin{equation}
    SDAM = \sum_{i=1}^{N} (x_i - \bar{x})^2
    \end{equation}
    \begin{equation}
    SDCM = \sum_{j=1}^{k} \sum_{i \in C_j} (x_i - \bar{x}_j)^2
    \end{equation}
    ただし、$N$ はデータ総数(グリッド数)、$k$ はクラス数、$x_i$ は個々のデータ値(各グリッド内の計測点数)、$\bar{x}$ は全データの平均値、$C_j$ は第 $j$ クラスに含まれるデータの集合、$\bar{x}_j$ は第 $j$ クラスの平均値を示す。    
    これにより、操業密度が高い海域と、使用頻度の低い海域を統計的に有意な基準で区分し、漁業者の経験的な認識と合致した可視化を実現する。

\end{enumerate}


\subsection{操業効率および燃油効率の定義}
本研究では、洋上風車の建設が漁業に与える影響を多角的に検証するため、物理的な移動負担を示す「単位移動距離あたりの漁獲量(CPUE)」と、投入エネルギーに対する生産性を示す「燃油効率($E_o$)」の 2 つの指標を採用する。

両指標を併用する理由は2つある。第一に、風車建設に伴う迂回行動や漁場探索の変化はまず「移動距離」の増加として現れるため、CPUE は空間的な影響を直接的に捉える感度の高い指標となる。第二に、漁業経営においては燃油コストが主要な支出を占めるため、$E_o$ を用いることで移動距離の変化が実際の経済効率に及ぼす影響を定量化できる。

以下に各指標の定義を示す。

\begin{enumerate}
    \item \textbf{航海距離の算出} \\
    1 回の航海における総移動距離は、GNSS 航跡データに含まれる時系列の計測点間の距離を累積して算出した。緯度・経度情報から 2 点間の距離 $D$ を求めるにあたっては、地球を球体と近似し、以下の Haversine の公式を用いた。
    \begin{equation}
    D = 2R \arcsin \left( \sqrt{\sin^2 \left( \frac{\phi_2 - \phi_1}{2} \right) + \cos(\phi_1) \cos(\phi_2) \sin^2 \left( \frac{\lambda_2 - \lambda_1}{2} \right)} \right)
    \end{equation}
    ここで、$R$ は地球の半径(6371 km)、$\phi_1, \phi_2$ は 2 点の緯度、$\lambda_1, \lambda_2$ は 2 点の経度である。

    \item \textbf{評価指標の定義} \\
    本研究では、操業の効率性を多角的に評価するため、以下の 2 つの指標を定義した。

    \begin{itemize}
        \item \textbf{単位努力量あたりの漁獲量(CPUE)} \\
        一般的に延縄漁業の CPUE 算出には「釣り針数」が努力量として用いられるが、本研究では前述した移動コストの変化を評価するため、1 回の航海における総移動距離 $D$ を漁獲努力量(Effort)と定義した。これに基づき、単位移動距離あたりの漁獲量を本研究における CPUE として次式で算出した。
        \begin{equation}
        \mathrm{CPUE} = \frac{W}{D}
        \end{equation}
        ここで、$W$ は当該航海における漁獲重量(水揚げ記録および操業日誌より取得)である。

        \item \textbf{燃油効率(Fuel Efficiency Index, $E_o$)} \\
        消費燃油量あたりの漁獲量として定義した。本研究で用いる給油履歴は、航海ごとではなく一定期間ごとの給油量 $O_{\mathrm{total}}$ として記録されているため、同期間内の漁獲総量 $W_{\mathrm{total}}$ を用いて期間全体の効率として算出した。
        \begin{equation}
        E_o = \frac{W_{\mathrm{total}}}{O_{\mathrm{total}}}
        \end{equation}
    \end{itemize}
\end{enumerate}