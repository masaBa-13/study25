\chapter{提案手法}
本章では、本研究の対象フィールドである北海道松前町沿岸の海域特性および対象漁業の概要を述べるとともに、解析に用いたデータセットの仕様、およびデータの統合・可視化の手法について記述する。
特に、複数のデータソース(GNSS航跡データ、水揚げ記録、操業日誌データ、給油記録)を統合し、マグロ延縄漁業の操業実態を抽出する手順、および比較手法について詳述する。

\section{対象フィールドと漁業概要}
本研究の対象フィールドは、北海道南端に位置する松前町沿岸である。この海域は対馬海流が流入する津軽海峡に面しており、マグロをはじめとする豊かな水産資源に恵まれた好漁場である。
対象とする漁業は、松前さくら漁業協同組合に加入する漁業者によるマグロ延縄漁業である。同漁業協同組合は、令和5年現在で正組合員217名が所属しており、各種網漁業、延縄漁業、釣り漁業、採介藻漁業および養殖漁業を複合的に営んでいる。その中でマグロ漁業の水揚げ金額は、全体の7.7\%を占めている。
本研究では、これらの漁業者が使用する漁船29隻(2024年時点)を分析対象とした。

\section{使用データセット}
本研究では、漁船の動静、漁獲実態、および操業効率を定量化するために、表\ref{tab:data_period}に示す4種類のデータを統合して使用した。

\begin{table}[h!]
    \centering
    \caption{使用データセットの名称と収録期間}
    \label{tab:data_period}
    \begin{tabular}{|l|l|} \hline
        データ名称 & 収録期間 \\ \hline \hline
        GNSS 航跡データ & 2023年7月 $\sim$ 2025年11月末 \\
        水揚げ記録 & 2023年7月 $\sim$ 2025年11月末 \\
        操業日誌データ & 2024年7月 $\sim$ 2025年11月末 \\ 
        給油履歴 & 2022年12月末 $\sim$ 2025年11月末 \\
        \hline
    \end{tabular}
\end{table}

\begin{figure}[h!]
    \centering
    \includegraphics[width=12cm]{images/period.png}
    \caption{使用データセットの収録期間}
    \label{fig:data_period}
\end{figure}

各データセットの詳細について以下に説明する。

\subsection{GNSS航跡データ}
対象漁船に搭載されたVMS(Vessel Monitoring System)から取得された航跡データである。
\begin{itemize}
    \item \textbf{サンプリング間隔}: 30秒
    \item \textbf{収録項目}: 漁船ID(PK)、日時、緯度、経度、速度、進行方向など
\end{itemize}

\subsection{水揚げ記録}
松前さくら漁協で管理しているマグロの水揚げデータである。
\begin{itemize}
    \item \textbf{対象魚種}: クロマグロ
    \item \textbf{収録項目}: 漁業者名(PK)、水揚げ日、マグロ重量など
\end{itemize}
本データは実際に水揚げされた正確な重量を示すデータとして機能するが、操業した上で漁獲できなかった日の情報が含まれないという制約がある。

\subsection{操業日誌データ}
漁業者がスマートフォン上の操業日誌アプリに入力したデータである。利用しているアプリは「マグログ」(2024年度に公立はこだて未来大学プロジェクト学習「スマート道南」チーム開発)\cite{maglog}であり、漁業者が操業終了後に日誌情報を入力し、クラウド上に保存する仕組みとなっている。
\begin{itemize}
    \item \textbf{収録項目}: 漁船名(PK)、操業日、海区番号、サイズ区分ごとの漁獲数など
    \item \textbf{サイズ区分}: $\sim$30kg, $\sim$50kg, $\sim$75kg, $\sim$100kg, 100kg$\sim$
\end{itemize}

本研究においてこのデータは、GNSS航跡データがマグロの漁獲を行っている時のものかどうかを判断するために使用する。
松前町のマグロ漁業者は、マグロ漁期であっても他の魚種を漁獲している場合がある。GNSS 航跡データ単体では何を漁獲していたかを判別することは困難である。また、3.2.2 の水揚げ記録だけでは、マグロ狙いで出漁したが漁獲が無かった日を把握できない。操業日誌データには、漁獲が無かった場合でも漁業者が操業記録を残しているため、これをGNSS 航跡データと照合することで、当該操業がマグロ延縄漁であっ
たことを特定することが可能となる。

なお、操業日誌のサイズ区分および入力値は漁業者の目測に基づくものであるため、漁協で記録されている水揚げ記録と比較して重量のズレが生じる可能性がある。この信憑性を検証するため、2024年7月から2025年11月までの期間において、同一の「日付」および「船ID」で記録され、かつ双方の報告尾数が完全に一致した1,676件を対象として、アプリ入力区分ごとの実績重量の分布を調査した(図\ref{fig:data_gosa})。
各入力区分における実績重量の分布を箱ひげ図を用いて確認した結果、全ての区分において、データの主要な分布を示す四分位範囲が、それぞれの正解重量範囲の内部に概ね収まっていることが確認された。「30〜50kg」区分においては、分布の上側(50kg以上)への若干の逸脱は見られるものの、箱の大部分は適正な範囲内に位置しており、漁業者による目測判定が高い精度で行われていることを示唆している。

一方で、四分位範囲の1.5倍を基準として外れ値を検出した結果、全1,676件中25件(全体の約1.5\%)の外れ値が確認された。これらは、入力区分と実績重量が大きく乖離している事例(例:$\sim$30kg区分に対し50kg以上の入力値など)であり、入力時の押し間違い等のヒューマンエラーに起因するものと考えられる。

以上の結果より、一部に誤入力や目測の誤差は含まれるものの、データ全体としては実態を正確に反映しており、本研究の分析に用いる上で十分な信憑性を有していると判断した。

\begin{figure}[h!]
    \centering
    \includegraphics[width=12cm]{images/hakohige.png}
    \caption{アプリ入力区分ごとの実績重量の分布}
    \label{fig:data_gosa}
\end{figure}

\subsection{給油履歴}
松前さくら漁協で管理している給油履歴である。
\begin{itemize}
    \item \textbf{収録項目}: 漁船名(PK)、給油日、油種(A重油または軽油)、給油量
    \item \textbf{データの性質}: 本データは毎回の操業ごとの給油を計測したものではなく、給油のタイミングで記録されたものである。本研究では29隻の対象船のうち、松前さくら漁協を通じて給油を行っており、給油記録が存在する27隻のデータを使用した。
\end{itemize}

\section{データ統合と前処理手法}
収集されたデータは形式や記録頻度が異なるため、Pythonを用いて統合処理を行った。
本研究では、GNSS航跡データに対する速度制限や移動距離、位置飛びに基づく機械的なフィルタリング処理は適用せず、水揚げ記録または操業日誌データとの照合による抽出処理のみを行った。一般的な漁船航跡解析では、異常値を除外するために厳格なフィルタリングを行うことがあるが、本研究対象海域においては、通信インフラとして利用しているLPWA(Low Power Wide Area)ネットワークの通信エリアに地理的な制約が存在した。具体的には、沿岸から離れた沖合海域や、岬の陰となる海域においてGNSS航跡データの受信頻度が低下し、データが断続的になる傾向が見られた。

この状況下でデータ送信間隔などのフィルタリング処理を適用した場合、本来有効であるはずの操業データまでもがノイズとして過剰に棄却され、解析可能なデータ数が著しく減少することが予備解析により確認された。実際、2024年のGNSS航跡データに対してフィルタリング処理(位置飛び1マイル以下、総移動距離3マイル以上等の条件)を適用してマグログデータと結合した結果、元データ約79万件に対し、最終的な抽出データは60,541件(約7.6\%)まで減少した。 そのため本研究では、通信環境に起因するデータの不連続性を許容し、可能な限り多くの操業実態を捕捉することを優先して、機械的なフィルタリングを行わない方針を採用した。

具体的には、目的に応じて以下の2種類の航跡データセットを構築した。

\subsection{操業日誌データに基づく航跡データセット}
操業日誌データとGNSS航跡データを照合し、両データ間で漁船名および日付が一致するレコードのみを抽出して構築したデータセットである。
これにより、漁獲が全く無かった日を含む、出漁した全操業の航跡が網羅されている。
\begin{itemize}
  \item \textbf{対象期間}: 2024年、2025年
  \item \textbf{特徴}: 漁獲の有無に関わらず全ての航跡が含まれるため、努力量が投入されたものの成果に結びつかなかった海域も含めた、正確な海域利用実態を可視化できる。
\end{itemize}

\subsection{水揚げ記録に基づく航跡データセット}
漁協が管理する水揚げ記録を参照し、「対象船29隻のうち、少なくとも1隻以上にマグロの水揚げがあった日」を特定し、その該当日における全対象船のGNSS航跡データを抽出して構築したデータセットである。

従来の手法では、水揚げ実績と航跡を1対1で照合するのが一般的であるが、本研究では漁業者へのヒアリング調査により得られた「ある一隻がマグロ漁に出ている日は、他の船も同様にマグロ漁を行っている可能性が高い」という意見に基づき、この手法を採用した。
これにより、実際に水揚げを記録した船だけでなく、同日に出漁したものの漁獲に至らなかった船の操業も含めて抽出している。
本研究の主眼である「操業実態の可視化」に関しては、3.3.1で述べた操業日誌データに基づく航跡データセットを用いることでより詳細に達成可能である。しかし、あえて本データセットを構築・併用する目的は以下の2点にある。

\begin{itemize}
  \item \textbf{簡易手法の精度検証}: 
  操業日誌データと比較することで、水揚げ記録のみに依存した手法が、実際の海域利用をどの程度の精度で再現できるか、あるいはどの程度の操業を取りこぼしていたかを定量的に評価するため。
  
  \item \textbf{長期間のトレンド解析}: 
  アプリ導入前(2023年)には操業日誌データが存在しないため、2023年から2025年にかけた3カ年にわたる漁場利用の経年変化の比較を行うためには、共通して利用可能な水揚げ記録に基づく本手法を用いる必要があるため。
\end{itemize}


\section{構築されたデータセットの概要}
\label{sec:dataset_overview}

\subsection{基本統計量}
3.3の手法に基づき構築されたデータセットの基本統計量を表\ref{tab:dataset_stats}に示す。
解析対象期間は各年の7月から翌1月のマグロ漁期であり、総データ件数は約64万レコードである。

注目すべき点は、「1日あたりの平均操業隻数」に顕著な違いが見られることである。
2024年を例に取ると、水揚げ記録ベースでは平均11.88隻であるのに対し、操業日誌ベースでは平均5.82隻となっている。
この数値の乖離は、3.3.2で述べたデータ抽出ロジックの差異に直接的に起因している。 水揚げ記録に基づくデータセットは、漁業者へのヒアリングに基づき「1隻でも漁獲があれば、その日の全漁船の航跡を抽出する」という拡張的な処理を行っているため、出漁日における船団全体の動き(漁獲が無かった船も含む)がカウントされ、結果として1日あたりの隻数が高く算出される。対して、操業日誌データに基づくデータセットは、「日付と漁船IDが共に一致したレコードのみを抽出する」という厳密な照合を行っているため、実際にアプリで操業を報告した個々の漁船のみがカウント対象となる。この定義の違いが平均値の差として表れていると考えられる。

\begin{table}[htbp]
  \centering
  \caption{構築されたデータセットの基本統計量}
  \label{tab:dataset_stats}
  \begin{tabular}{lcccccr}
    \hline
    データセット & 年度 & 期間 & 操業 & 参加 & 平均 & データ \\
     & & & 日数 & 船数 & 隻数/日 & 件数 \\
    \hline
    \textbf{水揚げ記録} & 2023 & 7/04 $\sim$ 12/13 & 38 & 26 & 5.63 & 89,640 \\
     & 2024 & 7/09 $\sim$ 11/10 & 26 & 29 & 11.88 & 175,987 \\
     & 2025 & 7/03 $\sim$ 11/27 & 31 & 30 & 13.19 & 166,921 \\
    \hline
    \textbf{操業日誌} 
    & 2023 & - & - & - & - & - \\
    & 2024 & 7/09 $\sim$ 11/15 & 28 & 29 & 5.82 & 103,698 \\
    & 2025 & 7/03 $\sim$ 11/27 & 33 & 29 & 8.18 & 108,102 \\
    \hline
  \end{tabular}
\end{table}

\subsection{データセット間の網羅性と差異}
アプリ導入の効果とデータの正確性を検証するため、2024年〜2025年シーズンにおける「水揚げ記録」と「操業日誌データ」の日付レベルでの突合を行った結果を以下に示す。

\begin{enumerate}
  \item \textbf{ゼロキャッチデータの捕捉(操業日誌データのみに存在する日付)} \\
  両データを比較した結果、操業日誌データにのみ存在し、水揚げ記録には存在しない日付が計17日確認された。
  具体的には、2024年の8月29日、9月(2, 5, 13, 16, 17, 28日)、10月(1, 6, 17日)などが該当する。
  これらは、出漁したものの漁獲がなく、水揚げ記録として残らなかった操業日である。
  この結果は、本研究で構築した操業日誌データセットが、従来の手法では欠落していた操業を正確に捕捉できている可能性を示している。

  \item \textbf{アプリ入力の漏れ(水揚げ記録のみに存在する日付)} \\
  一方で、水揚げ記録にはあるが操業日誌には記録がない日付は、2024年10月16日および10月27日のわずか2日(全期間の約1.6\%)に留まった。
  これはアプリの入力漏れや通信エラーによるものが可能性として考えられるが、欠損率は極めて低く、本データセットが分析に耐えうる十分な信憑性と網羅性を有していると言える。
\end{enumerate}


\section{分析手法と評価指標}
\label{sec:visualization}
本節では、構築されたデータセットを用い、マグロ延縄漁業の操業実態を定量的に明らかにするための分析手法と評価指標について述べる。
なお、本研究における空間解析には、オープンソースの地理情報システム(GIS: Geographic Information System)である QGIS(Ver. 3.42.1-Münster)を使用した。

\subsection{GISを用いた空間利用の可視化}
    洋上風車の建設予定海域(促進区域)と漁業活動の空間的な重複状況を評価するため、以下の手順で漁船の利用頻度分布を可視化した。
\begin{enumerate}
    \item \textbf{六角形グリッドによる空間分割} \\
    分析対象海域を内接円の直径が1 km の六角形グリッドで分割し、空間データの集計単位とした(図\ref{fig:hex_grid})。
    \begin{figure}[h!]
        \centering
        \includegraphics[width=9cm]{images/hex.png}
        \caption{内接円直径1kmの六角形グリッド}
        \label{fig:hex_grid}
    \end{figure}
    六角形グリッドは、正方形グリッドと比較して隣接するセルの中心間距離が等しいため、移動体の経路や分布頻度を表現する際の方向依存性(異方性)が低く、空間的なバイアスを軽減できる利点がある。
    グリッドの生成には、QGIS の「グリッドを作成(Create Grid)」ツールを用い、グリッドタイプとして六角形(Polygon, Hexagonal Grid)を選択して作成した。

    \item \textbf{GNSS 航跡データの集計と利用頻度の定義} \\
    CSV 形式でインポートしたGNSS 航跡データの各計測点が、作成した六角形グリッドのいずれに属するかを特定し、各グリッドに含まれる計測点数(ポイント数)を算出した。この集計処理には、QGIS の解析ツールである「ポリゴン内のポイント数を数える(Count Points in Polygon)」を使用した。
    なお、本研究で使用するVMS端末の測位間隔は30秒で一定であるため、グリッド内の計測点数は当該海域における漁船の\textbf{滞在時間(漁獲努力量)}と等価であり、これを本研究における「利用頻度」と定義した。

    \item \textbf{分位数(Quantile)による階級区分} \\
    算出された利用頻度の可視化にあたり、階級区分には「分位数分類(Quantile Classification)」を採用し、全データを等量ずつ10階級(10\%刻み)に分割して色分けを行った。
    本研究で比較を行う「水揚げ記録ベース」と「操業日誌ベース」のデータセット間には、レコード数に最大で7万件ほど(約1.7倍)の大きな開きが存在する。絶対値に基づく分類を用いた場合、データ数の多い年度のみが高頻度と判定される恐れがあるため、分位数分類を用いることでデータ量の多寡に依存せず、各データセット内における相対的な主要漁場の位置関係を公平に比較可能とした。
\end{enumerate}

\subsection{操業効率および燃油効率の定義}
本研究では、洋上風車の建設が漁業に与える影響を多角的に検証するため、物理的な移動負担を示す「単位移動距離あたりの漁獲量(CPUE:Catch Per Unit Effort)」と、投入エネルギーに対する生産性を示す「燃油効率($E_o$)」の2つの指標を採用する。

両指標を併用する理由は2つある。第一に、洋上風車建設に伴う迂回行動や漁場の変化はまず「移動距離」の増加として現れるため、CPUEは空間的な影響を直接的に捉える感度の高い指標となる。第二に、漁業経営においては燃油コストが主要な支出を占めるため、$E_o$を用いることで移動距離の変化が実際の経済効率に及ぼす影響を定量化できる。

以下に各指標の定義を示す。

\begin{enumerate}
    \item \textbf{移動距離の算出} \\
    1回の操業における総移動距離は、GNSS航跡データに含まれる時系列の計測点間の距離を累積して算出した。緯度・経度情報から2点間の距離$D$を求めるにあたっては、地球を球体と近似し、以下のHaversineの公式を用いた。

    \begin{equation}
        D = 2R \arcsin \left( \sqrt{ \sin^2 \left( \frac{\phi_2 - \phi_1}{2} \right) + \cos \phi_1 \cos \phi_2 \sin^2 \left( \frac{\lambda_2 - \lambda_1}{2} \right) } \right)
        \label{eq:haversine}
    \end{equation}

    ここで、$R$は地球の半径(6,371 km)、$\phi_1, \phi_2$は2点の緯度、$\lambda_1, \lambda_2$は2点の経度である。

    \item \textbf{評価指標の定義}
    \begin{itemize}
        \item \textbf{単位努力量あたりの漁獲量(CPUE)} \\
        1 回の操業における総移動距離 $D$ を漁獲努力量(Effort)と定義し、次式で算出した。
        \begin{equation}
        \mathrm{CPUE} = \frac{W}{D}
        \end{equation}
        ここで、$W$ は当該操業における漁獲重量(水揚げ記録および操業日誌より取得)である。

        \item \textbf{燃油効率(Fuel Efficiency Index, $E_o$)} \\
        一定期間ごとの給油量 $O_{\mathrm{total}}$ と、同期間内の漁獲総量 $W_{\mathrm{total}}$ を用いて期間全体の効率として算出した。
        \begin{equation}
        E_o = \frac{W_{\mathrm{total}}}{O_{\mathrm{total}}}
        \end{equation}
    \end{itemize}
\end{enumerate}